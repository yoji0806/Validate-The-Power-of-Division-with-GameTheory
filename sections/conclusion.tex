\documentclass[main.tex]{subfiles}
\begin{document}



\section{終章}

設定に四苦八苦したモデルだったが、均衡が計算できた時はとても面白いおもちゃが作れたように感じた。
しかしまだ今モデルにおいても分析しきれていない点や、本来の疑問にモデルで答えきれていない部分があるので、今後の展望としてまとめる。

今後の展望は3つある。

1つ目は、別の均衡の計算である。今回は$T^*$が全て内点という条件での均衡を分析したが、いくつかのパラメーターがかなり制限されてしまった。
例えば、内閣の時間割引率$\delta_{内閣}$や、内閣の政策への偏り$\alpha, \beta$である。
これらは$T^*_{有事}, T^*_{平時}$の少なくとも一つが端点をとる場合を考えることで、より広い範囲を分析できる。

2つ目は、国会の暴走に注目したモデルの拡張である。今回のモデル設定では国会はタイプを2つで、その一つ(愚民:$\delta_L$)においては最適行動を仮定していた。
国会の暴走はポピュリズムを念頭に置いてモデルを設計したが、
この点について詳しく分析するには、$\delta_{市民}$を連続変数にするのが良いかもしれない。

3つ目は、司法をプレイヤーとして組み込むことである。
現実では、司法は合憲/違憲を判断するだけでなく、あえて判断しないという選択をすることで内閣の権力を認めるような振る舞いをする。
また日本では裁判長の個人色が薄い印象があるがアメリカでは濃い。
それは制度的な違いではなく、まだ日本ではたまたま発生していないだけかもしれない。これらの点がモデル化の鍵かもしれない。





\end{document}