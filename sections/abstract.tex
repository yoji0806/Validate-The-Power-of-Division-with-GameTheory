\documentclass[main.tex]{subfiles}
\begin{document}

\section*{要旨}

本研究は、三権分立のうち「立法(国会)」と「行政(内閣)」の権力分立構造がどのように権力の暴走を抑制し、
社会全体の利益を実現するかをゲーム理論の枠組みを用いて分析したものである。

日本を含む多くの先進国において三権分立制は当たり前のように存在するが、「なぜ大丈夫なのか」「どんなときにも権力の暴走は防げるのか」といった根本的な問いにはあまりされない。
さらに、教科書やモンテスキューやロックの古典的著作に当たっても、現状の制度や叙情的な説明が中心であり、網羅性や客観性に欠ける部分が残されている。
そこで本論文では、国家の方針決定や政策形成の過程を舞台として数理モデルを構築し、
権力分立が「権力の暴走」をどのような条件下で抑制あるいは助長するのかを明確にすることを目的とした。

本稿のモデルは、2期間の不完備情報ベイジアン動学ゲームによって構成されている。
プレイヤーは内閣と国会の2者のみとし、司法は組み込まない。
その理由としては、(1) 行政や立法のプロセス(政策・法律立案と承認)に司法が直接的に参加する場面が少ないこと、
(2) 違憲判決が多くの場合、実際の行政執行や立法から相当の時間を経て下されるため、当事者(政権や議席数)が変わった別のゲームに近い状況となること、
(3) 本モデルの主題である「有事・平時に外部から問題が降ってきたとき、国家がどのような意思決定を行うか」という場面では、違法・違憲行為の有無が主要な争点にはならないためである。

具体的には、「有事か平時か」という世界の状態と、「市民(主権者)の時間割引率」がプレイヤーごとに非対称に観測される設定を行い、
内閣・国会それぞれの効用関数を定義したうえでベイジアンナッシュ均衡を分析する。
内閣は自身の“私欲”(独自の正義感や政治的モチベーション)と公共心とのバランスをパラメータ$T$で表し、
国会は「市民の厚生」を直接代表するものとして、時間割引率の水準(市民が“賢民”か“愚民”か)を観測できるように設定する。
一方、国会の時間割引率や内閣の私欲パラメータは相手プレイヤーには観測されない。
これにより、「お互いが完全には見えていない」状況下でどのように政策が選択されるかを検討し、暴走の起こりやすさや抑制条件を解析した。

考察の結果、本モデルでは以下のような知見が得られた。
今回見た均衡では内閣の時間割引率$\delta_{内閣}$が賢民の時間割引率$\delta_H$よりも低くなる必要がある。
またこの内閣は実は有事であっても、非予防的政策(B)を選好する部分がある。
私欲が増せば増すほど、内閣は予防的政策(A)を主張しやすくなる。
これは、内点の仮定から生じた条件$\alpha>\beta$、つまり内閣は予防的政策(A)を世界の状況に関係なく選好していることが理由に挙げられる。
以上の分析を踏まえ、今後の展望としては、別の均衡の計算や、国の時間割引率をはじめとするパラメータの連続化などの拡張、司法の組み込みなどが挙げられる。


\end{document}