\documentclass[main.tex]{subfiles}
\begin{document}

\section{国会の最適反応}


\subsection{市民が愚民($\delta_L$)である場合}

仮定6と仮定7より、市民が愚民である場合、国会は有事でも平時でも即時的政策($B$)を選好する。
よって、愚民である場合は$a_{国会}=B$が支配戦略になる。

\proposition{$a^*_{国会}=B \quad\text{if}\quad \delta_{市民} = \delta_L $}


\subsection{市民が賢民($\delta_H$)である場合}

仮定5と仮定7より、市民が賢民である場合は、国会は有事には予防的政策($A$)を平時には即時的政策($B$)を選好する。
しかし、世界状況が有事か平時かは国会には分からない。内閣の行動を通じて合理的に予測する。

まず、内閣の行動に対する確率を以下のように一旦仮置きして計算を進める。
\begin{align*}
    & P(m_{内閣} = A| W = {有事}, T > T^*_{有事}) = e \\
    & P(m_{内閣} = B| W = {有事}, T > T^*_{有事} ) = 1-e \\
    & P(m_{内閣} = A| W = {有事}, T < T^*_{有事}) = f \\
    & P(m_{内閣} = B| W = {有事}, T < T^*_{有事} ) = 1-f \\
    & P(m_{内閣} = A| W = {平時}, T > T^*_{有事}) = g \\
    & P(m_{内閣} = B| W = {平時}, T > T^*_{有事} ) = 1-g \\
    & P(m_{内閣} = A| W = {平時}, T < T^*_{有事}) = h \\
    & P(m_{内閣} = B| W = {平時}, T < T^*_{有事} ) = 1-h
\end{align*}

国会が求めたいのは、内閣の先手の行動$m_{内閣}=\{A,B\}$を見た上での条件付き確率$P(W=有事|m_{内閣})$である\footnote{片方がわかれば、$P(W=平時|m_{内閣}) = 1-P(W=有事|m_{内閣})$でもう片方も計算できる。}。
さらに、次章で見るように内閣の最適行動は$T,T^*$の大小関係によっても決まるため、実際に計算しないといけないのは以下の確率である\footnote{$m_{内閣}$は$=\{A,B\}$でそれぞれ場合分けする必要がある。}
\footnote{最後の変換は、$(W=有事, T>T^*_{有事})$を1つの事象とみなして、ベイズの公式を使って計算できます。
具体的には、
\begin{align*}
    P(m_{内閣}|W=有事, T>T^*_{有事})
    &= \frac{ P(m_{内閣}, W=有事, T>T^*_{有事}) }{ P(W=有事, T>T^*_{有事}) }\\
    &= \frac{ P(W=有事, T>T^*_{有事},m_{内閣}) }{ P(W=有事, T>T^*_{有事}) } \quad {より、}\\
\end{align*}
\begin{align*}
    & P(m_{内閣}|W=有事, T>T^*_{有事}) = \frac{ P(W=有事, T>T^*_{有事},m_{内閣}) }{ P(W=有事, T>T^*_{有事})}\\
    & P(W=有事, T>T^*_{有事},m_{内閣}) = P(m_{内閣}|W=有事, T>T^*_{有事}) × P(W=有事, T>T^*_{有事})\\
    & \quad {を使って、}\\
\end{align*}

\begin{align*}
    & P(W=有事, T>T^*_{有事}|m_{内閣}) \\
    &= \frac{ P(W=有事, T>T^*_{有事}, m_{内閣}) }{ P(m_{内閣}) }\\
    &= \frac{ P(m_{内閣}|W=有事, T>T^*_{有事}) × P(W=有事, T>T^*_{有事}) }{ P(m_{内閣}) }
\end{align*}
}。


\begin{align*}
    &  P(W=有事|m_{内閣})\\
    &= P(W=有事, T>T^*_{有事}|m_{内閣}) \;+\; P(W=有事, T<T^*_{有事}|m_{内閣})\\[0.5em]
    &= \frac{ P(W=有事, T>T^*_{有事}, m_{内閣}) }{ P(m_{内閣}) } + \frac{ P(W=有事, T<T^*_{有事},m_{内閣}) }{ P(m_{内閣}) }\\[1em]
    &= \frac{ P(m_{内閣} | W=有事, T>T^*_{有事}) × P(W=有事, T>T^*_{有事}) }{ P(m_{内閣}) }\\[0.5em]
    &\quad\quad + \frac{ P(m_{内閣} | W=有事, T<T^*_{有事}) × P(W=有事, T<T^*_{有事}) }{ P(m_{内閣}) }   
\end{align*}

最終的にこれを計算するために、以下、必要な確率を計算する。

xxx(TODO:ノート写す。)


\theendnotes
\end{document}