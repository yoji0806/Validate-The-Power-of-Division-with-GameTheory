\documentclass[main.tex]{subfiles}
\begin{document}

\section{国会の最適反応}


\subsection{市民が愚民($\delta_L$)である場合}

仮定6と仮定7より、市民が愚民である場合、国会は有事でも平時でも即時的政策($B$)を選好する。
つまり愚民である場合は$a_{国会}=B$が支配戦略になる。よって以下の命題を得る。

\proposition{$a^*_{国会}=B \quad\text{if}\quad \delta_{市民} = \delta_L $}


\subsection{市民が賢民($\delta_H$)である場合}

仮定5と仮定7より、市民が賢民である場合は、国会は有事には予防的政策($A$)を平時には即時的政策($B$)を選好する。
しかし、世界状況が有事か平時であるかは国会には分からない。国会は内閣の行動を通じて合理的に予測する。

まず、内閣の行動に対する確率を以下のように一旦仮置きして計算を進める。
\begin{align*}
    & P(m_{内閣} = A| W = {有事}, T > T^*_{有事}) = e \\
    & P(m_{内閣} = B| W = {有事}, T > T^*_{有事} ) = 1-e \\
    & P(m_{内閣} = A| W = {有事}, T < T^*_{有事}) = f \\
    & P(m_{内閣} = B| W = {有事}, T < T^*_{有事} ) = 1-f \\
    & P(m_{内閣} = A| W = {平時}, T > T^*_{有事}) = g \\
    & P(m_{内閣} = B| W = {平時}, T > T^*_{有事} ) = 1-g \\
    & P(m_{内閣} = A| W = {平時}, T < T^*_{有事}) = h \\
    & P(m_{内閣} = B| W = {平時}, T < T^*_{有事} ) = 1-h
\end{align*}

国会の最適反応を考えるにあたって必要なことは、国会の立場から見たときの、
先手である内閣の行動$m_{内閣}=\{A,B\}$を見た上での条件付き確率$P(W=有事|m_{内閣})$である
\footnote{
    有事、平時の確率がわかれば、それに対応する国会の最適行動が求まるから。また、条件付き期待値を考えるのは、$p,1-p$よりもより正確に有事,平時の確率を計算できるため。
    また有事か平時かの条件付き確率が片方わかれば、$P(W=平時|m_{内閣}) = 1-P(W=有事|m_{内閣})$でもう片方も計算できる。}。
さらに、次章で見るように内閣の最適行動は$T,T^*$の大小関係によっても決まるため、実際に計算しないといけないのは以下の確率である\footnote{$m_{内閣}$は$=\{A,B\}$でそれぞれ場合分けする必要がある。}
\footnote{最後の変換は、$(W=有事, T>T^*_{有事})$を1つの事象とみなして、ベイズの公式を使って計算できます。
具体的には、
\begin{align*}
    P(m_{内閣}|W=有事, T>T^*_{有事})
    &= \frac{ P(m_{内閣}, W=有事, T>T^*_{有事}) }{ P(W=有事, T>T^*_{有事}) }\\
    &= \frac{ P(W=有事, T>T^*_{有事},m_{内閣}) }{ P(W=有事, T>T^*_{有事}) } \quad {より、}\\
\end{align*}
\begin{align*}
    & P(m_{内閣}|W=有事, T>T^*_{有事}) = \frac{ P(W=有事, T>T^*_{有事},m_{内閣}) }{ P(W=有事, T>T^*_{有事})}\\
    & P(W=有事, T>T^*_{有事},m_{内閣}) = P(m_{内閣}|W=有事, T>T^*_{有事}) × P(W=有事, T>T^*_{有事})\\
    & \quad {を使って、}\\
\end{align*}
\begin{align*}
    & P(W=有事, T>T^*_{有事}|m_{内閣}) \\
    &= \frac{ P(W=有事, T>T^*_{有事}, m_{内閣}) }{ P(m_{内閣}) }\\
    &= \frac{ P(m_{内閣}|W=有事, T>T^*_{有事}) × P(W=有事, T>T^*_{有事}) }{ P(m_{内閣}) }
\end{align*}
}。


\begin{align*}
    &  P(W=有事|m_{内閣})\\
    &= P(W=有事, T>T^*_{有事}|m_{内閣}) \;+\; P(W=有事, T<T^*_{有事}|m_{内閣})\\[0.5em]
    &= \frac{ P(W=有事, T>T^*_{有事}, m_{内閣}) }{ P(m_{内閣}) } + \frac{ P(W=有事, T<T^*_{有事},m_{内閣}) }{ P(m_{内閣}) }\\[1em]
    &= \frac{ P(m_{内閣} | W=有事, T>T^*_{有事}) × P(W=有事, T>T^*_{有事}) }{ P(m_{内閣}) }\\[0.5em]
    &\quad\quad + \frac{ P(m_{内閣} | W=有事, T<T^*_{有事}) × P(W=有事, T<T^*_{有事}) }{ P(m_{内閣}) }   
\end{align*}

最終的にこれを計算するために、以下、必要な確率を求める。

定義より
$$P(W={有事}) = p, \quad P(W={平時}) = 1-p$$


$T \sim U(0,1)$より、$r_{有事} \equiv \int_0^{T^*_{有事}} f(T)dT \quad {とすると、}$
\begin{align*}
    P(T>T^*_{有事}) &= \int_{T^*_{有事}}^1 f(T)dT\\
    &= 1- \int_0^{T^*_{有事}} f(T)dT\\
    &= 1-r_{有事}\\
    \\
    P(T<T^*_{有事}) &= \int_0^{T^*_{有事}} f(T)dT = r_{有事}\\
    \\
    P(T>T^*_{平時}) &= 1- \int_0^{T^*_{平時}} f(T)dT =  1-r_{平時}\\
    \\
    P(T<T^*_{平時}) &= \int_0^{T^*_{平時}} f(T)dT = r_{平時}
\end{align*}

これを用いて、\footnote{TODO:余力あれば、この意味を確かめる。時系列的に独立ではないが、無相関??}
\begin{align*}
    P(W=有事, T>T^*_{有事}) &= p × (1-r_{有事})\\
    P(W=有事, T<T^*_{有事}) &= p × r_{有事}\\
    P(W=平時, T>T^*_{有事}) &= p × (1-r_{平時})\\
    P(W=平時, T<T^*_{有事}) &= p × r_{平時}\\
\end{align*}

これを用いて、\footnote{最初の変形は一見ややこしいが、「傘を持っていく確率」=「晴れの日に傘を持っていく確率」×「晴れの確率」+ $\cdots \quad$ 
というように、「ある行動の確率」 = 「条件付き行動の確率」 × 「その条件の事象の発生確率」を計算しているだけ。}
\begin{align*}
    & P(m_{内閣} = A) \\
    &=  P(m_{内閣} = A | W=有事, T>T^*_{有事})P(W=有事, T>T^*_{有事}) \\
    &\quad + P(m_{内閣} = A | W=有事, T<T^*_{有事})P(W=有事, T<T^*_{有事})\\
    &\quad + P(m_{内閣} = A | W=平時, T>T^*_{平時})P(W=平時, T>T^*_{平時})\\
    &\quad + P(m_{内閣} = A | W=平時, T<T^*_{平時})P(W=平時, T<T^*_{平時})\\
    &= e × p(1-r_{有事}) + f× pr_{有事} + g(1-p)(1-r_{平時}) + h(1-p)r_{平時}\\
    \\
    & P(m_{内閣} = B) \\
    &= (1-e) × p(1-r_{有事}) + (1-f)× pr_{有事} + (1-g)(1-p)(1-r_{平時}) + (1-h)(1-p)r_{平時}
\end{align*}

これを用いて、\footnote{計算の方法は上の注と同じで、$(W=有事, T>T^*_{有事})$を1つの事象とみなして、ベイズの公式を使って計算できる。}
\begin{align*}
    & P(W=有事, T>T^*_{有事} | m_{内閣}=A)\\[0.5em]
    &= \frac{ P(m_{内閣}=A| W={有事}, T>T^*_{有事}) × P(W={有事}, T>T^*_{有事}) }{ P(m_{内閣}=A) } \\[1em]
    &= \frac{ ep(1-r_{有事}) }{ ep(1-r_{有事}) + fpr_{有事} + g(1-p)(1-r_{平時}) + h(1-p)r_{平時} }
\end{align*}

\begin{align*}
    & P(W=有事, T<T^*_{有事} | m_{内閣}=A)\\[0.5em]
    &= \frac{ fpr_{有事} }{ ep(1-r_{有事}) + fpr_{有事} + g(1-p)(1-r_{平時}) + h(1-p)r_{平時} }
\end{align*}

\begin{align*}
    & P(W=平時, T>T^*_{平時} | m_{内閣}=A)\\[0.5em]
    &= \frac{ g(1-p)(1-r_{平時}) }{ ep(1-r_{有事}) + fpr_{有事} + g(1-p)(1-r_{平時}) + h(1-p)r_{平時} }
\end{align*}

\begin{align*}
    & P(W=平時, T<T^*_{平時} | m_{内閣}=A)\\[0.5em]
    &= \frac{ h(1-p)r_{平時} }{ ep(1-r_{有事}) + fpr_{有事} + g(1-p)(1-r_{平時}) + h(1-p)r_{平時} }
\end{align*}



\begin{align*}
    & P(W=有事, T>T^*_{有事} | m_{内閣}=B)\\[0.5em]
    &= \frac{ P(m_{内閣}=B| W={有事}, T>T^*_{有事}) × P(W={有事}, T>T^*_{有事}) }{ P(m_{内閣}=B) } \\[1em]
    &= \frac{ (1-e)p(1-r_{有事}) }{ (1-e)p(1-r_{有事}) + (1-f)pr_{有事} + (1-g)(1-p)(1-r_{平時}) + (1-h)(1-p)r_{平時} }
\end{align*}

\begin{align*}
    & P(W=有事, T<T^*_{有事} | m_{内閣}=B)\\[0.5em]
    &= \frac{ (1-f)pr_{有事} }{ (1-e)p(1-r_{有事}) + (1-f)pr_{有事} + (1-g)(1-p)(1-r_{平時}) + (1-h)(1-p)r_{平時} }
\end{align*}

\begin{align*}
    & P(W=平時, T>T^*_{平時} | m_{内閣}=B)\\[0.5em]
    &= \frac{ (1-g)(1-p)(1-r_{平時}) }{ (1-e)p(1-r_{有事}) + (1-f)pr_{有事} + (1-g)(1-p)(1-r_{平時}) + (1-h)(1-p)r_{平時} }
\end{align*}

\begin{align*}
    & P(W=平時, T<T^*_{平時} | m_{内閣}=B)\\[0.5em]
    &= \frac{ (1-h)(1-p)r_{平時} }{ (1-e)p(1-r_{有事}) + (1-f)pr_{有事} + (1-g)(1-p)(1-r_{平時}) + (1-h)(1-p)r_{平時} }
\end{align*}


これを用いて、最終的に求めたい$P(W|m_{内閣})$の確率が計算できる。
\begin{align*}
    & P(W=有事 | m_{内閣}=A)\\[0.5em]
    &= P(W=有事, T>T^*_{有事} | m_{内閣}=A) + P(W=有事, T<T^*_{有事} | m_{内閣}=A)\\[0.5em]
    &= \frac{ ep(1-r_{有事}) +  fpr_{有事}  }{ ep(1-r_{有事}) + fpr_{有事} + g(1-p)(1-r_{平時}) + h(1-p)r_{平時} }
\end{align*}

\begin{align*}
    & P(W=平時 | m_{内閣}=A)\\[0.5em]
    &= P(W=平時, T>T^*_{平時} | m_{内閣}=A) + P(W=平時, T<T^*_{平時} | m_{内閣}=A)\\[0.5em]
    &= \frac{  g(1-p)(1-r_{平時}) + h(1-p)r_{平時}  }{ ep(1-r_{有事}) + fpr_{有事} + g(1-p)(1-r_{平時}) + h(1-p)r_{平時} }
\end{align*}

\begin{align*}
    & P(W=有事 | m_{内閣}=B)\\[0.5em]
    &= P(W=有事, T>T^*_{有事} | m_{内閣}=B) + P(W=有事, T<T^*_{有事} | m_{内閣}=B)\\[0.5em]
    &= \frac{ (1-e)p(1-r_{有事}) +  (1-f)pr_{有事} }{ (1-e)p(1-r_{有事}) + (1-f)pr_{有事} + (1-g)(1-p)(1-r_{平時}) + (1-h)(1-p)r_{平時} }
\end{align*}

\begin{align*}
    & P(W=平時 | m_{内閣}=B)\\[0.5em]
    &= P(W=平時, T>T^*_{平時} | m_{内閣}=B) + P(W=平時, T<T^*_{平時} | m_{内閣}=B)\\[0.5em]
    &= \frac{ (1-g)(1-p)(1-r_{平時}) + (1-h)(1-p)r_{平時} }{ (1-e)p(1-r_{有事}) + (1-f)pr_{有事} + (1-g)(1-p)(1-r_{平時}) + (1-h)(1-p)r_{平時} }
\end{align*}

\bigskip
これで内閣の各行動に対する、国会の最適反応を求めることができる。
まず内閣が1期目で$m_{内閣}=A$を選択、つまり、予防的政策($A$)を主張した場合の国会の各行動の期待値は

\begin{align*}
    & E[u_{国会}(a_{国会}=A, \delta_{市民}=\delta_H) | m_{内閣} = A  ]\\[0.5em]
    &= P(W=有事 | m_{内閣}=A) × (V'_{A1} + \delta_H V'_{A2}) + P(W=平時 | m_{内閣}=A) × (V_{A1} + \delta_H V_{A2})\\[0.5em]
    &= \frac{ \{ep(1-r_{有事}) +  fpr_{有事}\}(V'_{A1} + \delta_H V'_{A2}) +  \{g(1-p)(1-r_{平時}) + h(1-p)r_{平時}\}(V_{A1} + \delta_H V_{A2})  }{ ep(1-r_{有事}) + fpr_{有事} + g(1-p)(1-r_{平時}) + h(1-p)r_{平時} }
\end{align*}

\begin{align*}
    & E[u_{国会}(a_{国会}=B, \delta_{市民}=\delta_H) | m_{内閣} = A  ]\\[0.5em]
    &= P(W=有事 | m_{内閣}=A) × (V'_{B1} + \delta_H V'_{B2}) + P(W=平時 | m_{内閣}=A) × (V_{B1} + \delta_H V_{B2})\\[0.5em]
    &= \frac{ \{ep(1-r_{有事}) +  fpr_{有事}\}(V'_{B1} + \delta_H V'_{B2}) +  \{g(1-p)(1-r_{平時}) + h(1-p)r_{平時}\}(V_{B1} + \delta_H V_{B2})  }{ ep(1-r_{有事}) + fpr_{有事} + g(1-p)(1-r_{平時}) + h(1-p)r_{平時} }
\end{align*}



\theendnotes
\end{document}