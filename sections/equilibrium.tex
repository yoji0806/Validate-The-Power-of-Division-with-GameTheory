\documentclass[main.tex]{subfiles}
\begin{document}

\section{均衡}

内閣の純粋戦略は以下の4つになる。\footnote{TODO:どうやって16個から減らしたかをかく。結局端点を見るので減らないが。}
\begin{align*}
    \begin{cases}
        (W=有事) かつ (T>T^*_{有事})  \rightarrow A\\
        (W=有事) かつ (T<T^*_{有事})  \rightarrow B\\
        (W=平時) かつ (T>T^*_{平時})  \rightarrow A\\
        (W=平時) かつ (T<T^*_{平時})   \rightarrow B\\
    \end{cases}
    \begin{cases}
        \rightarrow B\\
        \rightarrow A\\
        \rightarrow B\\
        \rightarrow A\\
    \end{cases}
    \begin{cases}
        \rightarrow A\\
        \rightarrow B\\
        \rightarrow B\\
        \rightarrow A\\
    \end{cases}
    \begin{cases}
        \rightarrow B\\
        \rightarrow A\\
        \rightarrow A\\
        \rightarrow B\\
    \end{cases}
\end{align*}

それぞれの戦略でベイジアンナッシュ均衡になる条件を求める。

\bigskip
\noindent
\underline{戦略1. (A,B,A,B)}

次の戦略を考える。
\begin{align*}
    \begin{cases}
        (W=有事) かつ (T>T^*_{有事})  \rightarrow A\\
        (W=有事) かつ (T<T^*_{有事})  \rightarrow B\\
        (W=平時) かつ (T>T^*_{平時})  \rightarrow A\\
        (W=平時) かつ (T<T^*_{平時})   \rightarrow B\\
    \end{cases}
\end{align*}

この時、仮置きしておいた内閣の行動の確率は$(e,f,g,h) = (1,0,1,0)$となる。
後手の国会の最適反応を明らかにするために、先手の内閣の行動で場合分けして考える。

\bigskip
まず、$T^*$が全て内点、つまり$0<T^*_{有事}<1,\quad 0<T^*_{平時}<1,$の場合を考える。\\
この時、有事-(i)かつ平時-(i)の条件の成立が必要であった。改めて書くと\\

有事-(i)$\Leftrightarrow 0<T^*_{有事}<1,\quad \alpha-\beta + V'_{B1}-V'_{A1} + \delta_{内閣}V'_{B2} - \delta_{内閣}V'_{A2} > 0$

平時-(i)$\Leftrightarrow 0<T^*_{平時}<1,\quad \alpha-\beta + V_{B1}-V_{A1} + \delta_{内閣}V_{B2} - \delta_{内閣}V_{A2} > 0$

\begin{align*}
    \text{有事-(i)かつ平時-(i)} \Leftrightarrow 
    \begin{cases}
        \beta < \alpha, \\
        0 \le \delta_{内閣} < \frac{V'_{B1}-V'_{A1}}{V'_{A2} - V'_{B2}}\\
        0 \le \delta_{内閣} < \frac{\alpha-\beta + V_{B1}-V_{A1}}{V_{A2} - V_{B2}} \quad&\text{if}\quad V_{B2} < V_{A2} \quad{かつ}\quad \frac{\alpha-\beta + V_{B1}-V_{A1}}{V_{A2} - V_{B2}}<1\\
        0 \le \delta_{内閣} \le 1 \quad&\text{if}\quad V_{B2} < V_{A2} \quad{かつ}\quad 1 \le \frac{\alpha-\beta + V_{B1}-V_{A1}}{V_{A2} - V_{B2}}\\
        0 \le \delta_{内閣} \le 1 \quad&\text{if}\quad V_{A2} < V_{B2}
    \end{cases}
\end{align*}


(i)$m_{内閣}=A$の場合:
\begin{align*}
    & E[u_{国会}(a_{国会}=A, \delta_{市民}=\delta_H) | m_{内閣} = A  ] - E[u_{国会}(a_{国会}=B, \delta_{市民}=\delta_H) | m_{内閣} = A  ]\\[1em]
    &= \frac{ \{ep(1-r_{有事}) +  fpr_{有事}\}(V'_{A1} -V'_{B1} + \delta_H V'_{A2} - \delta_H V'_{B2})  }{ ep(1-r_{有事}) + fpr_{有事} + g(1-p)(1-r_{平時}) + h(1-p)r_{平時} }\\[1em]
    &\quad + \frac{ \{g(1-p)(1-r_{平時}) + h(1-p)r_{平時}\}(V_{A1} - V_{B1} + \delta_H V_{A2} - \delta_H V_{B2} ) }{ ep(1-r_{有事}) + fpr_{有事} + g(1-p)(1-r_{平時}) + h(1-p)r_{平時} }\\[1em]
    &= \frac{ p(1-r_{有事}) (V'_{A1} -V'_{B1} + \delta_H V'_{A2} - \delta_H V'_{B2})  }{ p(1-r_{有事})  + (1-p)(1-r_{平時}) }\\[1em]
    &\quad + \frac{ (1-p)(1-r_{平時})(V_{A1} - V_{B1} + \delta_H V_{A2} - \delta_H V_{B2} ) }{ p(1-r_{有事}) +  (1-p)(1-r_{平時}) }
\end{align*}

また、$T^*$が全て内点にある時、$r_{有事}=T^*_{有事}, \quad r_{平時} = T^*_{平時}$となるので、

\begin{align*}
    &= \frac{ p(1-T^*_{有事}) (V'_{A1} -V'_{B1} + \delta_H V'_{A2} - \delta_H V'_{B2})  }{ p(1-T^*_{有事})  + (1-p)(1-T^*_{平時}) }\\[1em]
    &\quad + \frac{ (1-p)(1-T^*_{平時})(V_{A1} - V_{B1} + \delta_H V_{A2} - \delta_H V_{B2} ) }{ p(1-T^*_{有事}) +  (1-p)(1-T^*_{平時}) }
\end{align*}


分母は正になるので、分子だけを考える。
以下の定義を思い出すと、

\begin{definition} \Large$1 - T^*_{有事} = \frac{ \alpha-\beta }{ \alpha-\beta + V'_{B1}-V'_{A1} + \delta_{内閣}V'_{B2} - \delta_{内閣}V'_{A2} }$ \end{definition}

\begin{definition} \Large$1 - T^*_{平時} = \frac{ \alpha-\beta }{ \alpha-\beta + V_{B1}-V_{A1} + \delta_{内閣}V_{B2} - \delta_{内閣}V_{A2} }$ \end{definition}








\begin{align*}
    &p(1-T^*_{有事}) (V'_{A1} -V'_{B1} + \delta_H V'_{A2} - \delta_H V'_{B2})    +    (1-p)(1-T^*_{平時})(V_{A1} - V_{B1} + \delta_H V_{A2} - \delta_H V_{B2} ) \\[1em]
    &= p( \frac{ \alpha-\beta }{ \alpha-\beta + V'_{B1}-V'_{A1} + \delta_{内閣}V'_{B2} - \delta_{内閣}V'_{A2} } ) (V'_{A1} -V'_{B1} + \delta_H V'_{A2} - \delta_H V'_{B2})  \\
    &\quad    +    (1-p)(  \frac{ \alpha-\beta }{ \alpha-\beta + V_{B1}-V_{A1} + \delta_{内閣}V_{B2} - \delta_{内閣}V_{A2} } )(V_{A1} - V_{B1} + \delta_H V_{A2} - \delta_H V_{B2} ) \\[1em]
    &= (\alpha - \beta) p( \frac{ 1 }{ \alpha-\beta + V'_{B1}-V'_{A1} + \delta_{内閣}V'_{B2} - \delta_{内閣}V'_{A2} } ) (V'_{A1} -V'_{B1} + \delta_H V'_{A2} - \delta_H V'_{B2})  \\
    &\quad    +     (\alpha - \beta)(1-p)(  \frac{ 1 }{ \alpha-\beta + V_{B1}-V_{A1} + \delta_{内閣}V_{B2} - \delta_{内閣}V_{A2} } )(V_{A1} - V_{B1} + \delta_H V_{A2} - \delta_H V_{B2} ) \\[1em]
    &= (\alpha - \beta) p( \frac{ V'_{A1} -V'_{B1} + \delta_H V'_{A2} - \delta_H V'_{B2} }{ \alpha-\beta + V'_{B1}-V'_{A1} + \delta_{内閣}V'_{B2} - \delta_{内閣}V'_{A2} } )  \\[0.5em]
    &\quad    +     (\alpha - \beta)(1-p)(  \frac{ V_{A1} - V_{B1} + \delta_H V_{A2} - \delta_H V_{B2}  }{ \alpha-\beta + V_{B1}-V_{A1} + \delta_{内閣}V_{B2} - \delta_{内閣}V_{A2} } )\\[1em]
    &= (\alpha - \beta) \Big\{ \quad p( \frac{ V'_{A1} -V'_{B1} + \delta_H V'_{A2} - \delta_H V'_{B2} }{ \alpha-\beta + V'_{B1}-V'_{A1} + \delta_{内閣}V'_{B2} - \delta_{内閣}V'_{A2} } )  \\[0.5em]
    &\quad\quad\quad\quad\quad\quad    +     (1-p)(  \frac{ V_{A1} - V_{B1} + \delta_H V_{A2} - \delta_H V_{B2}  }{ \alpha-\beta + V_{B1}-V_{A1} + \delta_{内閣}V_{B2} - \delta_{内閣}V_{A2} } )\Big\}\\[1em]
    &= (\alpha - \beta)×\\
    &\quad \Big\{  p( \frac{ V'_{A1} -V'_{B1} + \delta_H (V'_{A2} - V'_{B2}) }{ \alpha-\beta + V'_{B1}-V'_{A1} + \delta_{内閣}(V'_{B2} - V'_{A2}) } )  
                    +(1-p)(  \frac{ V_{A1} - V_{B1} + \delta_H (V_{A2} - V_{B2})  }{ \alpha-\beta + V_{B1}-V_{A1} + \delta_{内閣}(V_{B2} - V_{A2}) } )\Big\}\\[1em]
    &= (\alpha - \beta)×\\
    &\quad \Big\{  -p( \frac{  V'_{B1} - V'_{A1} + \delta_H (V'_{B2} - V'_{A2}) }{ \alpha-\beta + V'_{B1}-V'_{A1} + \delta_{内閣}(V'_{B2} - V'_{A2}) } )  
                    -(1-p)(  \frac{ V_{B1} - V_{A1} + \delta_H (V_{B2} - V_{A2})  }{ \alpha-\beta + V_{B1}-V_{A1} + \delta_{内閣}(V_{B2} - V_{A2}) } )\Big\}\\[1em]
    &= (\beta - \alpha)×\\
    &\quad \Big\{  p( \frac{  V'_{B1} - V'_{A1} + \delta_H (V'_{B2} - V'_{A2}) }{ \alpha-\beta + V'_{B1}-V'_{A1} + \delta_{内閣}(V'_{B2} - V'_{A2}) } )  
                    +(1-p)(  \frac{ V_{B1} - V_{A1} + \delta_H (V_{B2} - V_{A2})  }{ \alpha-\beta + V_{B1}-V_{A1} + \delta_{内閣}(V_{B2} - V_{A2}) } )\Big\}\\[1em]
\end{align*}


\noindent
仮定5より、
\begin{align*}
    & V'_{B1}  + \delta_H V'_{B2} < V'_{A1}  + \delta_H V'_{A2}\\
    &V'_{B1} -V'_{A1} + \delta_H( V'_{B2} - V'_{A2}) < 0
\end{align*}

仮定7より、
\begin{align*}
    & V_{A1}  + \delta_{市民} V_{A2} < V_{B1} +  \delta_{市民} V_{B2}\\
    & -V_{B1} + V_{A1} + \delta_{市民} V_{A2} - \delta_{市民} V_{B2} < 0\\
    & V_{B1} - V_{A1} + \delta_{市民} V_{B2} - \delta_{市民} V_{A2}  > 0\\
    & V_{B1} - V_{A1} + \delta_{市民} (V_{B2} - V_{A2}) > 0\\
    & V_{B1} - V_{A1} + \delta_{H} (V_{B2} - V_{A2}) > 0\\
\end{align*}

有事-(i)より、$\alpha-\beta + V'_{B1}-V'_{A1} + \delta_{内閣}V'_{B2} - \delta_{内閣}V'_{A2} > 0$

平時-(i)より、$\alpha-\beta + V_{B1}-V_{A1} + \delta_{内閣}V_{B2} - \delta_{内閣}V_{A2} > 0$

有事-(i)かつ平時-(i)より、$\beta < \alpha$

となり、全ての項の正負が明らかになる。しかし、全体の正負は不明なので場合分けをして考える。
見やすくするために、以下のように変数を置く。

\begin{align*}
    V'_{B1} - V'_{A1} + \delta_H (V'_{B2} - V'_{A2}) &\equiv X' < 0\\
    \alpha-\beta + V'_{B1}-V'_{A1} + \delta_{内閣}(V'_{B2} - V'_{A2}) &\equiv Y' > 0\\
    V_{B1} - V_{A1} + \delta_H (V_{B2} - V_{A2}) &\equiv X > 0\\
    \alpha-\beta + V_{B1}-V_{A1} + \delta_{内閣}(V_{B2} - V_{A2}) &\equiv Y > 0\\[1em]
    \frac{X'}{Y'} < 0,\quad \frac{X}{Y} > 0
\end{align*}

すると差分は
\begin{align*}
    &= (\beta - \alpha) × \Big\{  p \Big( \frac{X'}{Y'} \Big) +(1-p) \Big(  \frac{X}{Y} \Big) \Big\}\\[1em]
    &= (\beta - \alpha) × \frac{1}{Y'Y} \Big\{  p \Big( X'Y \Big) +(1-p) \Big(  XY' \Big) \Big\}\\[1em]
    &= (\beta - \alpha) × \frac{1}{Y'Y} \Big\{  p X'Y  + XY' -p XY'  \Big\}\\[1em]
    &= (\beta - \alpha) × \frac{1}{Y'Y} \Big\{  p \Big( X'Y - XY' \Big) + XY'   \Big\}
\end{align*}

となり、$Y'Y>0$なので、$ p \Big( X'Y - XY' \Big) + XY'$の正負が、そのまま全体の正負となる。
$p \Big( X'Y - XY' \Big) + XY'>0$の場合とはつまり、
\begin{align*}
    & p \Big( X'Y - XY' \Big) + XY' > 0\\
    & p \Big( X'Y - XY' \Big)  > -  XY'\\
    & p \Big(XY' - X'Y  \Big)  <  XY'\\[0.5em]
    & XY'>0,\quad X'Y<0より、\quad XY' - X'Y>0なので、\\[1em]
    & p   <  \frac{XY'}{XY' - X'Y}
\end{align*}

であり、$p \Big( X'Y - XY' \Big) + XY'<0$の場合とは、$p   >  \frac{XY'}{XY' - X'Y}$である。


よって、$T^*$が全て内点の場合の、国会の期待値の差分は以下になる。
\begin{align*}
    \begin{cases}
        p < \frac{XY'}{XY' - X'Y} \quad\Rightarrow\quad 全体>0 &\quad\Rightarrow\quad a^*_{国会}(\delta_H|m_{内閣} = A) = A\\[0.5em]
        p > \frac{XY'}{XY' - X'Y} \quad\Rightarrow\quad 全体<0 &\quad\Rightarrow\quad a^*_{国会}(\delta_H|m_{内閣} = A) = B\\[0.5em] 
     \end{cases}
\end{align*}


\bigskip
(i)$m_{内閣}=B$の場合:
\begin{align*}
    & E[u_{国会}(a_{国会}=A, \delta_{市民}=\delta_H) | m_{内閣} = B  ] - E[u_{国会}(a_{国会}=B, \delta_{市民}=\delta_H) | m_{内閣} = B  ]\\[1em]
    &= \frac{ \{(1-e)p(1-r_{有事}) +  (1-f)pr_{有事}\}(V'_{A1} -V'_{B1} + \delta_H V'_{A2} - \delta_H V'_{B2})  }{ ep(1-r_{有事}) + fpr_{有事} + g(1-p)(1-r_{平時}) + h(1-p)r_{平時} }\\[1em]
    &\quad + \frac{ \{(1-g)(1-p)(1-r_{平時}) + (1-h)(1-p)r_{平時}\}(V_{A1} - V_{B1} + \delta_H V_{A2} - \delta_H V_{B2} ) }{ ep(1-r_{有事}) + fpr_{有事} + g(1-p)(1-r_{平時}) + h(1-p)r_{平時} }\\[1em]
    &= \frac{  pr_{有事}(V'_{A1} -V'_{B1} + \delta_H V'_{A2} - \delta_H V'_{B2})  }{ p(1-r_{有事}) +(1-p)(1-r_{平時})  }\\[1em]
    &\quad + \frac{ (1-p)r_{平時}(V_{A1} - V_{B1} + \delta_H V_{A2} - \delta_H V_{B2} ) }{ p(1-r_{有事}) + (1-p)(1-r_{平時})  }
\end{align*}

また、$T^*$が全て内点にある時、$r_{有事}=T^*_{有事}, \quad r_{平時} = T^*_{平時}$となるので、

\begin{align*}
    &= \frac{  pT^*_{有事}(V'_{A1} -V'_{B1} + \delta_H V'_{A2} - \delta_H V'_{B2})  }{ p(1-T^*_{有事}) +(1-p)(1-T^*_{平時})  }\\[1em]
    &\quad + \frac{ (1-p)T^*_{平時}(V_{A1} - V_{B1} + \delta_H V_{A2} - \delta_H V_{B2} ) }{ p(1-T^*_{有事}) + (1-p)(1-T^*_{平時})  }
\end{align*}

分母は正になるので、分子だけを考える。
\begin{align*}
    &pT^*_{有事}(V'_{A1} -V'_{B1} + \delta_H V'_{A2} - \delta_H V'_{B2}) + (1-p)T^*_{平時}(V_{A1} - V_{B1} + \delta_H V_{A2} - \delta_H V_{B2})\\[1em]
    &= p \frac{ V'_{B1} - V'_{A1} +\delta_{内閣}V'_{B2} - \delta_{内閣}V'_{A2} }{ \alpha-\beta + V'_{B1}-V'_{A1} + \delta_{内閣}V'_{B2} - \delta_{内閣}V'_{A2} } (V'_{A1} -V'_{B1} + \delta_H V'_{A2} - \delta_H V'_{B2})\\[1em]
    &\quad + (1-p) \frac{ V_{B1} - V_{A1} +\delta_{内閣}V_{B2} - \delta_{内閣}V_{A2} }{ \alpha-\beta + V_{B1}-V_{A1} + \delta_{内閣}V_{B2} - \delta_{内閣}V_{A2} }(V_{A1} - V_{B1} + \delta_H V_{A2} - \delta_H V_{B2})\\[1em]
    &= p \frac{ V'_{B1} - V'_{A1} +\delta_{内閣}(V'_{B2} - V'_{A2}) }{ \alpha-\beta + V'_{B1}-V'_{A1} + \delta_{内閣}(V'_{B2} - V'_{A2}) } (V'_{A1} -V'_{B1} + \delta_H (V'_{A2} - V'_{B2}))\\[1em]
    &\quad + (1-p) \frac{ V_{B1} - V_{A1} +\delta_{内閣}(V_{B2} - V_{A2}) }{ \alpha-\beta + V_{B1}-V_{A1} + \delta_{内閣}(V_{B2} - V_{A2}) }(V_{A1} - V_{B1} + \delta_H (V_{A2} - V_{B2}))\\[1em]
    &= -p \frac{ V'_{B1} - V'_{A1} +\delta_{内閣}(V'_{B2} - V'_{A2}) }{ \alpha-\beta + V'_{B1}-V'_{A1} + \delta_{内閣}(V'_{B2} - V'_{A2}) } (V'_{B1} -V'_{A1} + \delta_H (V'_{B2} - V'_{A2}))\\[1em]
    &\quad - (1-p) \frac{ V_{B1} - V_{A1} +\delta_{内閣}(V_{B2} - V_{A2}) }{ \alpha-\beta + V_{B1}-V_{A1} + \delta_{内閣}(V_{B2} - V_{A2}) }(V_{B1} - V_{A1} + \delta_H (V_{B2} - V_{A2}))\\[1em]
\end{align*}


同じく、$X', Y', X, Y$を用いて、
$$Y' -\alpha + \beta = V'_{B1}-V'_{A1} + \delta_{内閣}(V'_{B2} - V'_{A2})$$
$$Y  -\alpha + \beta = V_{B1}-V_{A1} + \delta_{内閣}(V_{B2} - V_{A2})$$
となることに注意すると、
\begin{align*}
    &= -p \frac{ Y' -\alpha + \beta }{ Y' } X'  -    (1-p) \frac{ Y  -\alpha + \beta }{ Y }X\\[0.5em]
    &= -p \frac{ X'Y' - X'(\alpha - \beta) }{ Y'}   -    (1-p) \frac{ XY - X(\alpha - \beta) }{ Y }\\[0.5em]
    &= -p \frac{ X'Y'Y - X'Y(\alpha - \beta) }{ Y'Y}   -    (1-p) \frac{ XY'Y - XY'(\alpha - \beta) }{Y' Y }\\[0.5em]
    &= p \frac{ X'Y(\alpha - \beta) - X'Y'Y  }{ Y'Y}   +   (1-p) \frac{ XY'(\alpha - \beta) - XY'Y }{Y' Y }\\[0.5em]
    &= p \frac{ X'Y(\alpha - \beta) - X'Y'Y  }{ Y'Y}   +   \frac{ XY'(\alpha - \beta) - XY'Y }{Y' Y } - p \frac{ XY'(\alpha - \beta) - XY'Y }{Y' Y }\\[0.5em]
    &= p \Big\{ \frac{ X'Y(\alpha - \beta) - X'Y'Y  }{ Y'Y} - \frac{ XY'(\alpha - \beta) - XY'Y }{Y' Y } \Big\}   +   \frac{ XY'(\alpha - \beta) - XY'Y }{Y' Y } \\[0.5em]
    &= \frac{1}{Y'Y} \Big\{ p \{ X'Y(\alpha - \beta) - X'Y'Y  - XY'(\alpha - \beta) - XY'Y  \}   +   XY'(\alpha - \beta) - XY'Y  \Big\}\\[0.5em]
\end{align*}


%&= \frac{1}{Y'Y} \Big\{ -p \{ X'Y'Y - X'Y(\alpha - \beta) \}   -    (1-p) \{ XY'Y - XY'(\alpha - \beta)\} \Big\} \\[0.5em]
%&= \frac{1}{Y'Y} \Big\{ -p X'Y'Y +p X'Y(\alpha - \beta)   -    \{XY'Y - XY'(\alpha - \beta)\} + p\{XY'Y - XY'(\alpha - \beta)\} \Big\} \\[0.5em]
%&= \frac{1}{Y'Y} \Big\{ -p X'Y'Y +p X'Y(\alpha - \beta)   -    XY'Y + XY'(\alpha - \beta) + pXY'Y - pXY'(\alpha - \beta) \Big\} \\[0.5em]









\newpage
WIPメモ
\begin{align*}
    \frac{ p( \alpha-\beta + V_{B1}-V_{A1} + \delta_{内閣}V_{B2} - \delta_{内閣}V_{A2} ) }{ (1-p)( \alpha-\beta + V'_{B1}-V'_{A1} + \delta_{内閣}V'_{B2} - \delta_{内閣}V'_{A2} ) }    >    - \frac{ V_{A1} - V_{B1} + \delta_H V_{A2} - \delta_H V_{B2} }{ V'_{A1} -V'_{B1} + \delta_H V'_{A2} - \delta_H V'_{B2} } \\[1em]
\end{align*}




\begin{definition} \Large$T^*_{有事} = \frac{ V'_{B1} - V'_{A1} +\delta_{内閣}V'_{B2} - \delta_{内閣}V'_{A2} }{ \alpha-\beta + V'_{B1}-V'_{A1} + \delta_{内閣}V'_{B2} - \delta_{内閣}V'_{A2} }$ \end{definition}

\begin{definition} \Large$1 - T^*_{有事} = \frac{ \alpha-\beta }{ \alpha-\beta + V'_{B1}-V'_{A1} + \delta_{内閣}V'_{B2} - \delta_{内閣}V'_{A2} }$ \end{definition}


\begin{definition} \Large$T^*_{平時} = \frac{ V_{B1} - V_{A1} +\delta_{内閣}V_{B2} - \delta_{内閣}V_{A2} }{ \alpha-\beta + V_{B1}-V_{A1} + \delta_{内閣}V_{B2} - \delta_{内閣}V_{A2} }$ \end{definition}

\begin{definition} \Large$1 - T^*_{平時} = \frac{ \alpha-\beta }{ \alpha-\beta + V_{B1}-V_{A1} + \delta_{内閣}V_{B2} - \delta_{内閣}V_{A2} }$ \end{definition}





\theendnotes
\end{document}