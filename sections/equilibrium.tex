\documentclass[main.tex]{subfiles}
\begin{document}

\section{均衡}

内閣の純粋戦略は以下の4つになる。\footnote{TODO:どうやって16個から減らしたかをかく。結局端点を見るので減らないが。}
\begin{align*}
    \begin{cases}
        (W=有事) かつ (T>T^*_{有事})  \rightarrow A\\
        (W=有事) かつ (T<T^*_{有事})  \rightarrow B\\
        (W=平時) かつ (T>T^*_{平時})  \rightarrow A\\
        (W=平時) かつ (T<T^*_{平時})   \rightarrow B\\
    \end{cases}
    \begin{cases}
        \rightarrow B\\
        \rightarrow A\\
        \rightarrow B\\
        \rightarrow A\\
    \end{cases}
    \begin{cases}
        \rightarrow A\\
        \rightarrow B\\
        \rightarrow B\\
        \rightarrow A\\
    \end{cases}
    \begin{cases}
        \rightarrow B\\
        \rightarrow A\\
        \rightarrow A\\
        \rightarrow B\\
    \end{cases}
\end{align*}

それぞれの戦略でベイジアンナッシュ均衡になる条件を求める。

\bigskip
\noindent
\underline{戦略1. (A,B,A,B)}

次の戦略を考える。
\begin{align*}
    \begin{cases}
        (W=有事) かつ (T>T^*_{有事})  \rightarrow A\\
        (W=有事) かつ (T<T^*_{有事})  \rightarrow B\\
        (W=平時) かつ (T>T^*_{平時})  \rightarrow A\\
        (W=平時) かつ (T<T^*_{平時})   \rightarrow B\\
    \end{cases}
\end{align*}

この時、仮置きしておいた内閣の行動の確率は$(e,f,g,h) = (1,0,1,0)$となる。









\bigskip
TODO: 以下の続きを、国会の章にかく。
\begin{align*}
    P(T>T^*_{有事}) &= \int_1^{T^*_{有事}} f(T)dT\\
    &= \int_1^{T^*_{有事}} f(T)dT
\end{align*}


\end{document}