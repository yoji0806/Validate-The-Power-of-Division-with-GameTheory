\documentclass[main.tex]{subfiles}
\begin{document}

\section{均衡}

内閣の純粋戦略は以下の4つになる。\footnote{TODO:どうやって16個から減らしたかをかく。結局端点を見るので減らないが。}
\begin{align*}
    \begin{cases}
        (W=有事) かつ (T>T^*_{有事})  \rightarrow A\\
        (W=有事) かつ (T<T^*_{有事})  \rightarrow B\\
        (W=平時) かつ (T>T^*_{平時})  \rightarrow A\\
        (W=平時) かつ (T<T^*_{平時})   \rightarrow B\\
    \end{cases}
    \begin{cases}
        \rightarrow B\\
        \rightarrow A\\
        \rightarrow B\\
        \rightarrow A\\
    \end{cases}
    \begin{cases}
        \rightarrow A\\
        \rightarrow B\\
        \rightarrow B\\
        \rightarrow A\\
    \end{cases}
    \begin{cases}
        \rightarrow B\\
        \rightarrow A\\
        \rightarrow A\\
        \rightarrow B\\
    \end{cases}
\end{align*}

それぞれの戦略でベイジアンナッシュ均衡になる条件を求める。

\bigskip
\noindent
\underline{戦略1. (A,B,A,B)}

次の戦略を考える。
\begin{align*}
    \begin{cases}
        (W=有事) かつ (T>T^*_{有事})  \rightarrow A\\
        (W=有事) かつ (T<T^*_{有事})  \rightarrow B\\
        (W=平時) かつ (T>T^*_{平時})  \rightarrow A\\
        (W=平時) かつ (T<T^*_{平時})   \rightarrow B\\
    \end{cases}
\end{align*}

この時、仮置きしておいた内閣の行動の確率は$(e,f,g,h) = (1,0,1,0)$となる。
後手の国会の最適反応を明らかにするために、先手の内閣の行動で場合分けして考える。

(i)$m_{内閣}=A$の場合:
\begin{align*}
    & E[u_{国会}(a_{国会}=A, \delta_{市民}=\delta_H) | m_{内閣} = A  ] - E[u_{国会}(a_{国会}=B, \delta_{市民}=\delta_H) | m_{内閣} = A  ]\\[1em]
    &= \frac{ \{ep(1-r_{有事}) +  fpr_{有事}\}(V'_{A1} -V'_{B1} + \delta_H V'_{A2} - \delta_H V'_{B2})  }{ ep(1-r_{有事}) + fpr_{有事} + g(1-p)(1-r_{平時}) + h(1-p)r_{平時} }\\[1em]
    &\quad + \frac{ \{g(1-p)(1-r_{平時}) + h(1-p)r_{平時}\}(V_{A1} - V_{B1} + \delta_H V_{A2} - \delta_H V_{B2} ) }{ ep(1-r_{有事}) + fpr_{有事} + g(1-p)(1-r_{平時}) + h(1-p)r_{平時} }\\[1em]
    &= \frac{ p(1-r_{有事}) (V'_{A1} -V'_{B1} + \delta_H V'_{A2} - \delta_H V'_{B2})  }{ p(1-r_{有事})  + (1-p)(1-r_{平時}) }\\[1em]
    &\quad + \frac{ (1-p)(1-r_{平時})(V_{A1} - V_{B1} + \delta_H V_{A2} - \delta_H V_{B2} ) }{ p(1-r_{有事}) +  (1-p)(1-r_{平時}) }
\end{align*}









\end{document}