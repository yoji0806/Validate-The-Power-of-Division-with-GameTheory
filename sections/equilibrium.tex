\documentclass[main.tex]{subfiles}
\begin{document}

\section{均衡}

内閣の純粋戦略は以下の4つになる。\footnote{TODO:どうやって16個から減らしたかをかく。結局端点を見るので減らないが。}
\begin{align*}
    \begin{cases}
        (W=有事) かつ (T>T^*_{有事})  \rightarrow A\\
        (W=有事) かつ (T<T^*_{有事})  \rightarrow B\\
        (W=平時) かつ (T>T^*_{平時})  \rightarrow A\\
        (W=平時) かつ (T<T^*_{平時})   \rightarrow B\\
    \end{cases}
    \begin{cases}
        \rightarrow B\\
        \rightarrow A\\
        \rightarrow B\\
        \rightarrow A\\
    \end{cases}
    \begin{cases}
        \rightarrow A\\
        \rightarrow B\\
        \rightarrow B\\
        \rightarrow A\\
    \end{cases}
    \begin{cases}
        \rightarrow B\\
        \rightarrow A\\
        \rightarrow A\\
        \rightarrow B\\
    \end{cases}
\end{align*}

それぞれの戦略でベイジアンナッシュ均衡になる条件を求める。

\bigskip
\noindent
\underline{戦略1. (A,B,A,B)}

次の戦略を考える。
\begin{align*}
    \begin{cases}
        (W=有事) かつ (T>T^*_{有事})  \rightarrow A\\
        (W=有事) かつ (T<T^*_{有事})  \rightarrow B\\
        (W=平時) かつ (T>T^*_{平時})  \rightarrow A\\
        (W=平時) かつ (T<T^*_{平時})   \rightarrow B\\
    \end{cases}
\end{align*}

この時、仮置きしておいた内閣の行動の確率は$(e,f,g,h) = (1,0,1,0)$となる。
後手の国会の最適反応を明らかにするために、先手の内閣の行動で場合分けして考える。

(i)$m_{内閣}=A$の場合:
\begin{align*}
    & E[u_{国会}(a_{国会}=A, \delta_{市民}=\delta_H) | m_{内閣} = A  ] - E[u_{国会}(a_{国会}=B, \delta_{市民}=\delta_H) | m_{内閣} = A  ]\\[1em]
    &= \frac{ \{ep(1-r_{有事}) +  fpr_{有事}\}(V'_{A1} -V'_{B1} + \delta_H V'_{A2} - \delta_H V'_{B2})  }{ ep(1-r_{有事}) + fpr_{有事} + g(1-p)(1-r_{平時}) + h(1-p)r_{平時} }\\[1em]
    &\quad + \frac{ \{g(1-p)(1-r_{平時}) + h(1-p)r_{平時}\}(V_{A1} - V_{B1} + \delta_H V_{A2} - \delta_H V_{B2} ) }{ ep(1-r_{有事}) + fpr_{有事} + g(1-p)(1-r_{平時}) + h(1-p)r_{平時} }\\[1em]
    &= \frac{ p(1-r_{有事}) (V'_{A1} -V'_{B1} + \delta_H V'_{A2} - \delta_H V'_{B2})  }{ p(1-r_{有事})  + (1-p)(1-r_{平時}) }\\[1em]
    &\quad + \frac{ (1-p)(1-r_{平時})(V_{A1} - V_{B1} + \delta_H V_{A2} - \delta_H V_{B2} ) }{ p(1-r_{有事}) +  (1-p)(1-r_{平時}) }
\end{align*}

\bigskip
まず、$T^*$が全て内点、つまり$0<T^*_{有事}<1,\quad 0<T^*_{平時}<1,$の場合を考える。\\
この時、$r_{有事}=T^*_{有事}, \quad r_{平時} = T^*_{平時}$となるので、

\begin{align*}
    &= \frac{ p(1-T^*_{有事}) (V'_{A1} -V'_{B1} + \delta_H V'_{A2} - \delta_H V'_{B2})  }{ p(1-T^*_{有事})  + (1-p)(1-T^*_{平時}) }\\[1em]
    &\quad + \frac{ (1-p)(1-T^*_{平時})(V_{A1} - V_{B1} + \delta_H V_{A2} - \delta_H V_{B2} ) }{ p(1-T^*_{有事}) +  (1-p)(1-T^*_{平時}) }
\end{align*}


仮定5より、$0 < V'_{A1} -V'_{B1} + \delta_H V'_{A2} - \delta_H V'_{B2} $ \\
仮定7より、$0 > V_{A1} - V_{B1} + \delta_H V_{A2} - \delta_H V_{B2}$である。

分母は正になるので、分子だけを考えると、

\begin{align*}
    &p(1-T^*_{有事}) (V'_{A1} -V'_{B1} + \delta_H V'_{A2} - \delta_H V'_{B2})    +    (1-p)(1-T^*_{平時})(V_{A1} - V_{B1} + \delta_H V_{A2} - \delta_H V_{B2} ) \\[1em]
    &= p( \frac{ \alpha-\beta }{ \alpha-\beta + V'_{B1}-V'_{A1} + \delta_{内閣}V'_{B2} - \delta_{内閣}V'_{A2} } ) (V'_{A1} -V'_{B1} + \delta_H V'_{A2} - \delta_H V'_{B2})  \\
    &\quad    +    (1-p)(  \frac{ \alpha-\beta }{ \alpha-\beta + V_{B1}-V_{A1} + \delta_{内閣}V_{B2} - \delta_{内閣}V_{A2} } )(V_{A1} - V_{B1} + \delta_H V_{A2} - \delta_H V_{B2} ) \\[1em]
    &= (\alpha - \beta) p( \frac{ 1 }{ \alpha-\beta + V'_{B1}-V'_{A1} + \delta_{内閣}V'_{B2} - \delta_{内閣}V'_{A2} } ) (V'_{A1} -V'_{B1} + \delta_H V'_{A2} - \delta_H V'_{B2})  \\
    &\quad    +     (\alpha - \beta)(1-p)(  \frac{ 1 }{ \alpha-\beta + V_{B1}-V_{A1} + \delta_{内閣}V_{B2} - \delta_{内閣}V_{A2} } )(V_{A1} - V_{B1} + \delta_H V_{A2} - \delta_H V_{B2} ) \\[1em]
    &= (\alpha - \beta) p( \frac{ V'_{A1} -V'_{B1} + \delta_H V'_{A2} - \delta_H V'_{B2} }{ \alpha-\beta + V'_{B1}-V'_{A1} + \delta_{内閣}V'_{B2} - \delta_{内閣}V'_{A2} } )  \\[0.5em]
    &\quad    +     (\alpha - \beta)(1-p)(  \frac{ V_{A1} - V_{B1} + \delta_H V_{A2} - \delta_H V_{B2}  }{ \alpha-\beta + V_{B1}-V_{A1} + \delta_{内閣}V_{B2} - \delta_{内閣}V_{A2} } )\\[1em]
    &= (\alpha - \beta) \Big\{ \quad p( \frac{ V'_{A1} -V'_{B1} + \delta_H V'_{A2} - \delta_H V'_{B2} }{ \alpha-\beta + V'_{B1}-V'_{A1} + \delta_{内閣}V'_{B2} - \delta_{内閣}V'_{A2} } )  \\[0.5em]
    &\quad\quad\quad\quad\quad\quad    +     (1-p)(  \frac{ V_{A1} - V_{B1} + \delta_H V_{A2} - \delta_H V_{B2}  }{ \alpha-\beta + V_{B1}-V_{A1} + \delta_{内閣}V_{B2} - \delta_{内閣}V_{A2} } )\Big\}\\[1em]
    &= (\alpha - \beta)×\\
    &\quad \Big\{  p( \frac{ V'_{A1} -V'_{B1} + \delta_H (V'_{A2} - V'_{B2}) }{ \alpha-\beta + V'_{B1}-V'_{A1} + \delta_{内閣}(V'_{B2} - V'_{A2}) } )  
                    +(1-p)(  \frac{ V_{A1} - V_{B1} + \delta_H (V_{A2} - V_{B2})  }{ \alpha-\beta + V_{B1}-V_{A1} + \delta_{内閣}(V_{B2} - V_{A2}) } )\Big\}\\[1em]
    &= (\alpha - \beta)×\\
    &\quad \Big\{  -p( \frac{  V'_{B1} - V'_{A1} + \delta_H (V'_{B2} - V'_{A2}) }{ \alpha-\beta + V'_{B1}-V'_{A1} + \delta_{内閣}(V'_{B2} - V'_{A2}) } )  
                    -(1-p)(  \frac{ V_{B1} - V_{A1} + \delta_H (V_{B2} - V_{A2})  }{ \alpha-\beta + V_{B1}-V_{A1} + \delta_{内閣}(V_{B2} - V_{A2}) } )\Big\}\\[1em]
    &= (\beta - \alpha)×\\
    &\quad \Big\{  p( \frac{  V'_{B1} - V'_{A1} + \delta_H (V'_{B2} - V'_{A2}) }{ \alpha-\beta + V'_{B1}-V'_{A1} + \delta_{内閣}(V'_{B2} - V'_{A2}) } )  
                    +(1-p)(  \frac{ V_{B1} - V_{A1} + \delta_H (V_{B2} - V_{A2})  }{ \alpha-\beta + V_{B1}-V_{A1} + \delta_{内閣}(V_{B2} - V_{A2}) } )\Big\}\\[1em]
\end{align*}


これが、正になる条件は、
\begin{align*}
    &p( \frac{ \alpha-\beta }{ \alpha-\beta + V'_{B1}-V'_{A1} + \delta_{内閣}V'_{B2} - \delta_{内閣}V'_{A2} } ) (V'_{A1} -V'_{B1} + \delta_H V'_{A2} - \delta_H V'_{B2})  \\
    &\quad    +    (1-p)(  \frac{ \alpha-\beta }{ \alpha-\beta + V_{B1}-V_{A1} + \delta_{内閣}V_{B2} - \delta_{内閣}V_{A2} } )(V_{A1} - V_{B1} + \delta_H V_{A2} - \delta_H V_{B2} )  > 0\\[1em]
\end{align*}


$T^*$がどちらも内点を取る場合は、$\alpha - \beta > 0$なので、
\begin{align*}
    &p( \frac{ 1 }{ \alpha-\beta + V'_{B1}-V'_{A1} + \delta_{内閣}V'_{B2} - \delta_{内閣}V'_{A2} } ) (V'_{A1} -V'_{B1} + \delta_H V'_{A2} - \delta_H V'_{B2})  \\
    &\quad    +    (1-p)(  \frac{ 1 }{ \alpha-\beta + V_{B1}-V_{A1} + \delta_{内閣}V_{B2} - \delta_{内閣}V_{A2} } )(V_{A1} - V_{B1} + \delta_H V_{A2} - \delta_H V_{B2} )  > 0\\[1em]
\end{align*}

$T^*$がどちらも内点を取る場合は、$T^*の分母 > 0$なので、
\begin{align*}
    &p(  \alpha-\beta + V_{B1}-V_{A1} + \delta_{内閣}V_{B2} - \delta_{内閣}V_{A2} ) (V'_{A1} -V'_{B1} + \delta_H V'_{A2} - \delta_H V'_{B2})  \\
    &\quad    +    (1-p)( \alpha-\beta + V'_{B1}-V'_{A1} + \delta_{内閣}V'_{B2} - \delta_{内閣}V'_{A2} )(V_{A1} - V_{B1} + \delta_H V_{A2} - \delta_H V_{B2} )  > 0\\[1em]
\end{align*}

以下、$V_{A2}, V_{B2}$の大小関係で







WIPメモ
\begin{align*}
    \frac{ p( \alpha-\beta + V_{B1}-V_{A1} + \delta_{内閣}V_{B2} - \delta_{内閣}V_{A2} ) }{ (1-p)( \alpha-\beta + V'_{B1}-V'_{A1} + \delta_{内閣}V'_{B2} - \delta_{内閣}V'_{A2} ) }    >    - \frac{ V_{A1} - V_{B1} + \delta_H V_{A2} - \delta_H V_{B2} }{ V'_{A1} -V'_{B1} + \delta_H V'_{A2} - \delta_H V'_{B2} } \\[1em]
\end{align*}




\begin{definition} \Large$T^*_{有事} = \frac{ V'_{B1} - V'_{A1} +\delta_{内閣}V'_{B2} - \delta_{内閣}V'_{A2} }{ \alpha-\beta + V'_{B1}-V'_{A1} + \delta_{内閣}V'_{B2} - \delta_{内閣}V'_{A2} }$ \end{definition}

\begin{definition} \Large$1 - T^*_{有事} = \frac{ \alpha-\beta }{ \alpha-\beta + V'_{B1}-V'_{A1} + \delta_{内閣}V'_{B2} - \delta_{内閣}V'_{A2} }$ \end{definition}


\begin{definition} \Large$T^*_{平時} = \frac{ V_{B1} - V_{A1} +\delta_{内閣}V_{B2} - \delta_{内閣}V_{A2} }{ \alpha-\beta + V_{B1}-V_{A1} + \delta_{内閣}V_{B2} - \delta_{内閣}V_{A2} }$ \end{definition}

\begin{definition} \Large$1 - T^*_{平時} = \frac{ \alpha-\beta }{ \alpha-\beta + V_{B1}-V_{A1} + \delta_{内閣}V_{B2} - \delta_{内閣}V_{A2} }$ \end{definition}





\theendnotes
\end{document}