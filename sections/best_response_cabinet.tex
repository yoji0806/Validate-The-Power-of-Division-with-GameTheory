\documentclass[main.tex]{subfiles}
\begin{document}

\section{内閣の最適反応}



内閣の戦略のパターン数は、私的情報の$W=\lbrace 有事, 平時\rbrace, T\in[0,1]$によって決まる。
私欲パラメーターの$T$は連続変数であるため、内閣にとって$m_{内閣}=A,m_{内閣}=B$が無差別となる境目$T^*$が存在する。
実際の$T$が$T^*$よりも大きいか小さいかで、内閣の最適反応は変化する。
$T^*$は、世界の状態$W$に依存するため、$T^*_{有事}, T^*_{平時}$が存在する。
よって内閣の戦略は、世界の状態$W$と、$T$と$T^* = \lbrace T^*_{有事}, T^*_{平時} \rbrace$の大小関係によって戦略は場合分けされる。
具体的には、以下の4パターンでそれぞれ予防的政策($A$)を主張するか、非予防的政策($B$)を主張するかを選択する。
\begin{align*}
    & (W=有事) かつ (T > T^*_{有事}) の場合 \\
    & (W=有事) かつ (T < T^*_{有事}) の場合 \\
    & (W=平時) かつ (T > T^*_{平時}) の場合 \\
    & (W=平時) かつ (T < T^*_{平時}) の場合 
\end{align*}


内閣の最適戦略は国会の最適戦略に依存する。しかし、国会の最適戦略も同じように内閣の最適戦略に依存している。
そこで、国会の行動に対する確率を以下のように一旦仮置きして計算を進める。
\begin{align*}
    & P(a_{国会} = A| m_{内閣} = A, \delta_{市民} = \delta_H ) \equiv x \\
    & P(a_{国会} = B| m_{内閣} = A, \delta_{市民} = \delta_H ) \equiv 1-x \\
    & P(a_{国会} = A| m_{内閣} = B, \delta_{市民} = \delta_H ) \equiv y \\
    & P(a_{国会} = B| m_{内閣} = B, \delta_{市民} = \delta_H ) \equiv 1-y
\end{align*}

市民が愚民である($\delta_{市民}=\delta_L$)場合、補題1より、国会は内閣の行動に依存せず、常に非予防的政策($B$)を選択する。
そのため、ここでは市民が賢民である($\delta_{市民}=\delta_H$)場合について考えれば良い。




\subsection{各行動が無差別となる境界点$T^*$の数式表現}


まず、内閣にとっての各行動が無差別になる境目である$T^* = \lbrace T^*_{有事}, T^*_{平時} \rbrace$を求める。
$T^*_{有事}$は、有事において内閣の期待効用が無差別となる状態である。有事における各行動の期待効用は
\begin{align*}
    & E[u_{内閣}(m_{内閣}=A|W=有事)]\\
    & = q\Big\lbrace 
        x \lbrace T\alpha + (1-T)(V'_{A1} + \delta_{内閣} V'_{A2}) \rbrace + (1-x)\lbrace T\beta + (1-T)(V'_{B1} + \delta_{内閣} V'_{B2}) \rbrace 
        \Big\rbrace\\
    & \quad +(1-q)\lbrace T\beta + (1-T)(V'_{B1} + \delta_{内閣} V'_{B2}) \rbrace
\end{align*}
\begin{align*}
    & E[u_{内閣}(m_{内閣}=B|W=有事)]\\
    & = q\Big\lbrace
        y \lbrace T\alpha + (1-T)(V'_{A1} + \delta_{内閣} V'_{A2}) \rbrace + (1-y)\lbrace T\beta + (1-T)(V'_{B1} + \delta_{内閣} V'_{B2}) \rbrace 
        \Big\rbrace\\
    & \quad +(1-q)\lbrace T\beta + (1-T)(V'_{B1} + \delta_{内閣} V'_{B2}) \rbrace
\end{align*}

\noindent
この差分は、
\begin{align*}
    & E[u_{内閣}(m_{内閣}=A|W=有事)] - E[u_{内閣}(m_{内閣}=B|W=有事)]\\
    & = q\Big\lbrace
        T\alpha(x-y) + V'_{A1}(x-y) + \delta_{内閣}V'_{A2}(x-y) - TV'_{A1}(x-y) - T\delta_{内閣}V'_{A2}(x-y) \\
    &   \quad\quad -T\beta(x-y) - V'_{B1}(x-y) - \delta_{内閣}V'_{B2}(x-y) + TV'_{B1}(x-y) + T\delta_{内閣}V'_{B2}(x-y)
        \Big\rbrace\\
    & = q(x-y) \lbrace T(\alpha-\beta + V'_{B1}-V'_{A1} + \delta_{内閣}V'_{B2} - \delta_{内閣}V'_{A2}) 
                +V'_{A1} - V'_{B1} + \delta_{内閣}(V'_{A2} - V'_{B2}) \rbrace
\end{align*}

\noindent
この差分が$=0$になる時が、2つの行動が無差別になる状態$T^*_{有事}$なので
\begin{gather*}
    E[u_{内閣}(m_{内閣}=A|W=有事)] - E[u_{内閣}(m_{内閣}=B|W=有事)] = 0 \\
    q(x-y) \lbrace T(\alpha-\beta + V'_{B1}-V'_{A1} + \delta_{内閣}V'_{B2} - \delta_{内閣}V'_{A2}) 
                +V'_{A1} - V'_{B1} + \delta_{内閣}(V'_{A2} - V'_{B2})  \rbrace = 0 \\
    T(\alpha-\beta + V'_{B1}-V'_{A1} + \delta_{内閣}V'_{B2} - \delta_{内閣}V'_{A2}) 
    = V'_{B1} - V'_{A1} +\delta_{内閣}V'_{B2} - \delta_{内閣}V'_{A2}
\end{gather*}
よって、
$$T = \frac{ V'_{B1} - V'_{A1} +\delta_{内閣}V'_{B2} - \delta_{内閣}V'_{A2} }{ \alpha-\beta + V'_{B1}-V'_{A1} + \delta_{内閣}V'_{B2} - \delta_{内閣}V'_{A2} }
= T^*_{有事}$$


次に$T^*_{平時}$も同じように求める。平時における各行動の期待値の差分は、
\begin{align*}
    & E[u_{内閣}(m_{内閣}=A|W=平時)] - E[u_{内閣}(m_{内閣}=B|W=平時)]\\
    & = q\Big\lbrace
        x \lbrace T\alpha + (1-T)(V_{A1} + \delta_{内閣} V_{A2}) \rbrace + (1-x)\lbrace T\beta + (1-T)(V_{B1} + \delta_{内閣} V_{B2}) \rbrace 
        \Big\rbrace\\
    & \quad +(1-q)\lbrace T\beta + (1-T)(V_{B1} + \delta_{内閣} V_{B2}) \rbrace\\
    & \quad - q\Big\lbrace
        y \lbrace T\alpha + (1-T)(V_{A1} + \delta_{内閣} V_{A2}) \rbrace + (1-y)\lbrace T\beta + (1-T)(V_{B1} + \delta_{内閣} V_{B2}) \rbrace
        \Big\rbrace\\
    & \quad +(1-q)\lbrace T\beta + (1-T)(V_{B1} + \delta_{内閣} V_{B2}) \rbrace\\
    & = qx \lbrace T\alpha + V_{A1} + \delta_{内閣}V_{A2} - TV_{A1} - T\delta_{内閣}V_{A2} - T\beta -V_{B1} -\delta_{内閣}V_{B2} + TV_{B1} + T\delta_{内閣}V_{B2} \rbrace\\
    & \quad -qy\lbrace T\alpha + V_{A1} + \delta_{内閣}V_{A2} - TV_{A1} - T\delta_{内閣}V_{A2} - T\beta -V_{B1} -\delta_{内閣}V_{B2} + TV_{B1} + T\delta_{内閣}V_{B2} \rbrace\\
    & = q(x-y)\lbrace T(\alpha -V_{A1} -\delta_{内閣}V_{A2} - \beta + V_{B1} + \delta_{内閣}V_{B2}) + V_{A1} + \delta_{内閣}V_{A2} - V_{B1} - \delta_{内閣}V_{B2}  \rbrace
\end{align*}

\noindent
この差分が$=0$になる時が、2つの行動が無差別になる状態$T^*_{平時}$なので
\begin{gather*}
    T(\alpha -V_{A1} -\delta_{内閣}V_{A2} - \beta + V_{B1} + \delta_{内閣}V_{B2}) + V_{A1} + \delta_{内閣}V_{A2} - V_{B1} - \delta_{内閣}V_{B2} = 0\\
\end{gather*}
よって、
$$T = \frac{ V_{B1} - V_{A1} +\delta_{内閣}V_{B2} - \delta_{内閣}V_{A2} }{ \alpha-\beta + V_{B1}-V_{A1} + \delta_{内閣}V_{B2} - \delta_{内閣}V_{A2} }
= T^*_{平時}$$


まとめると、$T^* = \lbrace T^*_{有事}, T^*_{平時} \rbrace$の数式上の定義は以下になる。

\vspace{0.5\baselineskip}

\begin{definition} \Large$T^*_{有事} = \frac{ V'_{B1} - V'_{A1} +\delta_{内閣}V'_{B2} - \delta_{内閣}V'_{A2} }{ \alpha-\beta + V'_{B1}-V'_{A1} + \delta_{内閣}V'_{B2} - \delta_{内閣}V'_{A2} }$ \end{definition}

\vspace{0.5\baselineskip}

\begin{definition} \Large$T^*_{平時} = \frac{ V_{B1} - V_{A1} +\delta_{内閣}V_{B2} - \delta_{内閣}V_{A2} }{ \alpha-\beta + V_{B1}-V_{A1} + \delta_{内閣}V_{B2} - \delta_{内閣}V_{A2} }$ \end{definition}





\subsection{$T^*$ の性質}

TODO:差分を一回微分してどうなるか。






\subsection{$T^*$ が全て内点に存在する条件}

内閣の私欲パラメーターである$T$は$[0,1]$の範囲に存在する。
しかし、$T^* = \lbrace T^*_{有事}, T^*_{平時} \rbrace$はそうとは限らない。
$0<T^*<1$のように内点に存在する場合は、最適反応が変わる境界点としての役割を持つ。
しかし$T^*<0,1<T^*$のように端点に存在する場合は、$T$の値によって最適反応は変わらない\footnote{TODO:=がついたときも考える or ここでは考えない言い訳をする。}。
よって、$T^*$が内点に存在する場合と、端点に存在する場合に分けて考える必要がある。
以下では、まず$T^*$ が全て内点に存在する場合の条件を明示する。


$W=有事$の場合、$T^*_{有事} = \frac{ V'_{B1} - V'_{A1} +\delta_{内閣}V'_{B2} - \delta_{内閣}V'_{A2} }{ \alpha-\beta + V'_{B1}-V'_{A1} + \delta_{内閣}V'_{B2} - \delta_{内閣}V'_{A2} }$
となる。
内点の条件は$0<T^*<1$だが、$T^*_{有事}, T^*_{平時}$の分母の正負で場合分けが必要。

\bigskip
\noindent
\textbf{有事-(i)}\; $0<T^*_{有事}<1,\;\;\alpha-\beta + V'_{B1}-V'_{A1} + \delta_{内閣}V'_{B2} - \delta_{内閣}V'_{A2} > 0$

まず分母の条件を整理すると、
\begin{align*}
    \delta_{内閣}(V'_{B2} - V'_{A2}) &> -\alpha+\beta - V'_{B1}+V'_{A1} \\
    \delta_{内閣}(V'_{A2} - V'_{B2}) &< \alpha-\beta + V'_{B1}-V'_{A1}\\
    \delta_{内閣} &< \frac{\alpha-\beta + V'_{B1}-V'_{A1}}{V'_{A2} - V'_{B2}} 
\end{align*}

$0<T^*_{有事}$より、
\begin{align*}
    0 &< \frac{ V'_{B1} - V'_{A1} +\delta_{内閣}V'_{B2} - \delta_{内閣}V'_{A2} }{ \alpha-\beta + V'_{B1}-V'_{A1} + \delta_{内閣}V'_{B2} - \delta_{内閣}V'_{A2} }\\
    0 &<  V'_{B1} - V'_{A1} +\delta_{内閣}(V'_{B2} - V'_{A2})\\
    -V'_{B1} + V'_{A1} &< \delta_{内閣}(V'_{B2} - V'_{A2})\\
    V'_{B1} - V'_{A1} &> \delta_{内閣}(V'_{A2} - V'_{B2})\\
    \delta_{内閣} &< \frac{V'_{B1} - V'_{A1}}{V'_{A2} - V'_{B2}}
\end{align*}


$T^*_{有事}<1$より、
\begin{align*}
    \frac{ V'_{B1} - V'_{A1} +\delta_{内閣}V'_{B2} - \delta_{内閣}V'_{A2} }{ \alpha-\beta + V'_{B1}-V'_{A1} + \delta_{内閣}V'_{B2} - \delta_{内閣}V'_{A2} } &< 1\\
    V'_{B1} - V'_{A1} +\delta_{内閣}V'_{B2} - \delta_{内閣}V'_{A2} &< \alpha-\beta + V'_{B1}-V'_{A1} + \delta_{内閣}V'_{B2} - \delta_{内閣}V'_{A2}\\
    \delta_{内閣}(V'_{B2} - V'_{A2}) &< \alpha-\beta + \delta_{内閣}(V'_{B2} - V'_{A2})\\
    \beta &< \alpha
\end{align*}

この3つの条件と$0 \le \delta_{内閣} \le 1$を合わせて、以下の最終的な条件を得る。
\begin{align*}
    \text{有事-(i)} \Leftrightarrow 
    \begin{cases}
        \beta < \alpha, \\
        0 \le \delta_{内閣} < \frac{V'_{B1}-V'_{A1}}{V'_{A2} - V'_{B2}}
    \end{cases}
\end{align*}



\bigskip
\noindent
\textbf{有事-(ii)}\; $0<T^*_{有事}<1,\;\; \alpha-\beta + V'_{B1}-V'_{A1} + \delta_{内閣}V'_{B2} - \delta_{内閣}V'_{A2} < 0$

まず分母の条件より
$$\delta_{内閣} > \frac{\alpha-\beta + V'_{B1}-V'_{A1}}{V'_{A2} - V'_{B2}}$$

$0<T^*_{有事}$より、
$$\delta_{内閣} > \frac{V'_{B1} - V'_{A1}}{V'_{A2} - V'_{B2}}$$

$T^*_{有事}<1$より、
$$\beta > \alpha$$

この3つの条件と$0 \le \delta_{内閣} \le 1$を合わせて、以下の最終的な条件を得る。
\begin{align*}
    \text{有事-(ii)} \Leftrightarrow 
    \begin{cases}
        \alpha < \beta, \\
        \frac{V'_{B1}-V'_{A1}}{V'_{A2} - V'_{B2}} < \delta_{内閣} \le 1
    \end{cases}
\end{align*}


\bigskip
\noindent
\textbf{平時-(i)}\; $0<T^*_{平時}<1,\;\; \alpha-\beta + V_{B1}-V_{A1} + \delta_{内閣}V_{B2} - \delta_{内閣}V_{A2} > 0$

まず分母の条件を整理すると、
\begin{align*}
    \delta_{内閣}(V_{B2} - V_{A2}) > -\alpha+\beta - V_{B1}+V_{A1} \\
    \delta_{内閣}(V_{A2} - V_{B2}) < \alpha-\beta + V_{B1}-V_{A1}\\
    \begin{cases}
        \delta_{内閣} &< \frac{\alpha-\beta + V_{B1}-V_{A1}}{V_{A2} - V_{B2}} \quad\text{if}\quad V_{B2} < V_{A2}\\
        \delta_{内閣} &> \frac{\alpha-\beta + V_{B1}-V_{A1}}{V_{A2} - V_{B2}} \quad\text{if}\quad V_{A2} < V_{B2}
    \end{cases}
\end{align*}

$0<T^*_{平時}$より、
\begin{align*}
    0 < \frac{ V_{B1} - V_{A1} +\delta_{内閣}V_{B2} - \delta_{内閣}V_{A2} }{ \alpha-\beta + V_{B1}-V_{A1} + \delta_{内閣}V_{B2} - \delta_{内閣}V_{A2} }\\
    0 <  V_{B1} - V_{A1} +\delta_{内閣}(V_{B2} - V_{A2})\\
    -V_{B1} + V_{A1} < \delta_{内閣}(V_{B2} - V_{A2})\\
    V_{B1} - V_{A1} > \delta_{内閣}(V_{A2} - V_{B2})\\
    \delta_{内閣}(V_{A2} - V_{B2}) < V_{B1} - V_{A1}\\
    \begin{cases}
        \delta_{内閣} &< \frac{V_{B1} - V_{A1}}{V_{A2} - V_{B2}} \quad\text{if}\quad V_{B2} < V_{A2}\\
        \delta_{内閣} &> \frac{V_{B1} - V_{A1}}{V_{A2} - V_{B2}} \quad\text{if}\quad V_{A2} < V_{B2}
      \end{cases}
\end{align*}


$T^*_{平時}<1$より、
\begin{align*}
    \frac{ V_{B1} - V_{A1} +\delta_{内閣}V_{B2} - \delta_{内閣}V_{A2} }{ \alpha-\beta + V_{B1}-V_{A1} + \delta_{内閣}V_{B2} - \delta_{内閣}V_{A2} } &< 1\\
    V_{B1} - V_{A1} +\delta_{内閣}V_{B2} - \delta_{内閣}V_{A2} &< \alpha-\beta + V_{B1}-V_{A1} + \delta_{内閣}V_{B2} - \delta_{内閣}V_{A2}\\
    \delta_{内閣}(V_{B2} - V_{A2}) &< \alpha-\beta + \delta_{内閣}(V_{B2} - V_{A2})\\
    \beta &< \alpha
\end{align*}


この3つの条件と$0 \le \delta_{内閣} \le 1$を合わせて、以下の最終的な条件を得る。
\begin{align*}
    \text{平時-(i)} \Leftrightarrow 
    \begin{cases}
        \beta < \alpha, \\
        0 \le \delta_{内閣} < \frac{\alpha-\beta + V_{B1}-V_{A1}}{V_{A2} - V_{B2}} \quad&\text{if}\quad V_{B2} < V_{A2} \quad{かつ}\quad \frac{\alpha-\beta + V_{B1}-V_{A1}}{V_{A2} - V_{B2}}<1\\
        0 \le \delta_{内閣} \le 1 \quad&\text{if}\quad V_{B2} < V_{A2} \quad{かつ}\quad 1 \le \frac{\alpha-\beta + V_{B1}-V_{A1}}{V_{A2} - V_{B2}}\\
        0 \le \delta_{内閣} \le 1 \quad&\text{if}\quad V_{A2} < V_{B2}
    \end{cases}
\end{align*}



\bigskip
\noindent
\textbf{平時-(ii)}\; $0<T^*_{平時}<1,\;\; \alpha-\beta + V_{B1}-V_{A1} + \delta_{内閣}V_{B2} - \delta_{内閣}V_{A2} < 0$

まず分母の条件より
\begin{align*}
    \delta_{内閣}(V_{B2} - V_{A2}) < -\alpha+\beta - V_{B1}+V_{A1} \\
    \delta_{内閣}(V_{A2} - V_{B2}) > \alpha-\beta + V_{B1}-V_{A1}\\
    \begin{cases}
        \delta_{内閣} &> \frac{\alpha-\beta + V_{B1}-V_{A1}}{V_{A2} - V_{B2}} \quad\text{if}\quad V_{B2} < V_{A2}\\
        \delta_{内閣} &< \frac{\alpha-\beta + V_{B1}-V_{A1}}{V_{A2} - V_{B2}} \quad\text{if}\quad V_{A2} < V_{B2}
    \end{cases}
\end{align*}


$0<T^*_{平時}$より、
\begin{align*}
    \begin{cases}
        \delta_{内閣} &> \frac{V_{B1} - V_{A1}}{V_{A2} - V_{B2}} \quad\text{if}\quad V_{B2} < V_{A2}\\
        \delta_{内閣} &< \frac{V_{B1} - V_{A1}}{V_{A2} - V_{B2}} \quad\text{if}\quad V_{A2} < V_{B2}
      \end{cases}
\end{align*}

$T^*_{平時}<1$より、
$$\beta > \alpha$$

この3つの条件を合わせて、以下の条件を得る。
\begin{align*}
    \text{平時-(ii)} \Leftrightarrow 
    \begin{cases}
        \alpha < \beta, \\
        \delta_{内閣} > \frac{V_{B1} - V_{A1}}{V_{A2} - V_{B2}} \quad&\text{if}\quad V_{B2} < V_{A2}\\
        \delta_{内閣} < \frac{\alpha-\beta + V_{B1}-V_{A1}}{V_{A2} - V_{B2}} \quad&\text{if}\quad V_{A2} < V_{B2}
    \end{cases}
\end{align*}

\noindent
条件の$V_{B2} < V_{A2}$と仮定1より$1 < \frac{V_{B1} - V_{A1}}{V_{A2} - V_{B2}}$ であること、\\
条件の$V_{A2} < V_{B2}$より$\frac{\alpha-\beta + V_{B1}-V_{A1}}{V_{A2} - V_{B2}} < 0$であることに注意する。\\
さらに$0 \le \delta_{内閣} \le 1$と合わせて、以下の最終的な条件を得る。
\begin{align*}
    \text{平時-(ii)} \Leftrightarrow 
    \begin{cases}
        \alpha < \beta, \\
        解なし \quad&\text{if}\quad V_{B2} < V_{A2}\\
        解なし \quad&\text{if}\quad V_{A2} < V_{B2}
    \end{cases}
\end{align*}

\noindent
つまり、平時-(ii) を満たす $\delta_{内閣}$は存在しない。

\bigskip
まとめると、有事と平時、またそれらの分母の正負から以下の3つの場合分けが生じる。
\begin{align*}
    \text{有事-(i)} \Leftrightarrow 
    \begin{cases}
        \beta < \alpha, \\
        0 \le \delta_{内閣} < \frac{V'_{B1}-V'_{A1}}{V'_{A2} - V'_{B2}}
    \end{cases}
    \text{有事-(ii)} \Leftrightarrow 
    \begin{cases}
        \alpha < \beta, \\
        \frac{V'_{B1}-V'_{A1}}{V'_{A2} - V'_{B2}} < \delta_{内閣} \le 1
    \end{cases}
\end{align*}

\begin{align*}
    \text{平時-(i)} \Leftrightarrow 
    \begin{cases}
        \beta < \alpha, \\
        0 \le \delta_{内閣} < \frac{\alpha-\beta + V_{B1}-V_{A1}}{V_{A2} - V_{B2}} \quad&\text{if}\quad V_{B2} < V_{A2} \quad{かつ}\quad \frac{\alpha-\beta + V_{B1}-V_{A1}}{V_{A2} - V_{B2}}<1\\
        0 \le \delta_{内閣} \le 1 \quad&\text{if}\quad V_{B2} < V_{A2} \quad{かつ}\quad 1 \le \frac{\alpha-\beta + V_{B1}-V_{A1}}{V_{A2} - V_{B2}}\\
        0 \le \delta_{内閣} \le 1 \quad&\text{if}\quad V_{A2} < V_{B2}
    \end{cases}
\end{align*}


\bigskip
$T^*$ が全て内点に存在するとは、有事-(i)かつ平時-(i)、有事-(ii)かつ平時-(i)のどちらかが成立している状態である。
それぞれの場合ごとに条件の成立可否と最終的な条件を出す。

\noindent
\begin{align*}
    \text{有事-(i)かつ平時-(i)} \Leftrightarrow 
    \begin{cases}
        \beta < \alpha, \\
        0 \le \delta_{内閣} < \frac{V'_{B1}-V'_{A1}}{V'_{A2} - V'_{B2}}\\
        0 \le \delta_{内閣} < \frac{\alpha-\beta + V_{B1}-V_{A1}}{V_{A2} - V_{B2}} \quad&\text{if}\quad V_{B2} < V_{A2} \quad{かつ}\quad \frac{\alpha-\beta + V_{B1}-V_{A1}}{V_{A2} - V_{B2}}<1\\
        0 \le \delta_{内閣} \le 1 \quad&\text{if}\quad V_{B2} < V_{A2} \quad{かつ}\quad 1 \le \frac{\alpha-\beta + V_{B1}-V_{A1}}{V_{A2} - V_{B2}}\\
        0 \le \delta_{内閣} \le 1 \quad&\text{if}\quad V_{A2} < V_{B2}
    \end{cases}
\end{align*}

\noindent
有事-(ii)かつ平時-(i):

$\beta < \alpha, \alpha < \beta$が矛盾するので不成立


\bigskip
よって、$T^* = \lbrace T^*_{有事}, T^*_{平時} \rbrace$ が全て内点に存在する場合、有事-(i)かつ平時-(i)の条件が必要になる。





\subsection{$T^*$ が全て端点に存在する条件}

その存在区間から導かれる条件を明らかにする。



\subsection{$T^*$ が内点と端点に存在する条件}

(i)有事で内点、平時で端点 $\Big\{  \big\{ \{\} \big\} \Big\} $

(ii)有事で端点、平時で内点$\lbrace asd   test asd \rbrace$


\theendnotes %章末注
\end{document}