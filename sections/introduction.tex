\documentclass[main.tex]{subfiles}
\begin{document}

\section*{序章}
\addcontentsline{toc}{section}{序章}
\setcounter{page}{1}

日本をはじめ先進国のほとんど全ての国家が三権分立制を採用している。
三権分立制とは平たくいえば、「ルールを決めるもの」「ルールを実行するもの」「ルールから外れていないか監視するもの」の3つの機関に国家権力を
分割して、互いに抑制させようという考えである。日本ではルールを決めるものを国会(立法)が相当し、実行するものは内閣(行政)が、監視するものは最高裁判所(司法)が
その役割を担っている。

多くの人が三権分立また権力分立という言葉を知ったのは中学・高校の社会科目であろう。例えば、(宮本憲一ほか, 2018)では以下のような記述がある。
\begin{quote}
  ロックは立法権と執行権を分離し、立法権優位の制度をとることを提案し、モンテスキューは『法の精神』で立法・行政・司法をわけ、異なる期間に担当させる
  三権分立制を唱えた。いずれも権力を複数の機関に分担させ、抑制と均衡(チェック-アンド-バランス)の関係におこうとした主張であった。
\end{quote}
多くの教科書ではこの導入部分の後に、各権力の働きや他の権力へ指名権や任命権、拒否権などを通じて各権力がどのように関わっているのか、
また、弾劾裁判などの問題が発生した場合の対処方法などを教わる。
これらに共通することは、権力分立はこうなっているという現状については知ることができるが、なぜ大丈夫なのか?という理由については答えてくれない。
本当にパワーバランスが均衡しているのか、どんな時でも権力の暴走は防がれるのか、それとも何か条件があるのか。その暴走は常に防がれるべき悪いものなのか、疑問は尽きない。
また、これらの議論は言葉を用いて行われることが多いが、客観性や議論の網羅性に欠ける可能性が高い。
モンテスキューの原著にもあたってみたが、叙情的な説明が多く、満足いく答えは得られなかった。

そこで今回、ゲーム理論を用いて権力分立のメカニズムを明らかにする。具体的には、上に挙げたような疑問に答えるため、政策・国家方針形成のプロセスを舞台として
内閣と国会を主役としたゲーム理論のモデルを作成し、どのような要素が暴走を抑制・助長するのか、その要素間の関係や効果を持つための条件を明らかにする。



\end{document}