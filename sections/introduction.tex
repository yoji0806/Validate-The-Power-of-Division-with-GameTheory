\documentclass[main.tex]{subfiles}
\begin{document}

\section*{序章}
\addcontentsline{toc}{section}{序章}
\setcounter{page}{1}

% \subsection*{研究動機}
\begin{itemize}
  \item 中学・高校の教科書には、三権分立しているから大丈夫(国家権力の暴走は防がれている)とある。政治学の教科書にもそうある(?)
  \item モンテスキューに遡っても、ルソーに遡っても、「こういうふうに分立したら大丈夫」とあるが、その理由やメカニズムは明らかになっていない。
  \item 今回見たいのは、各国家権力それぞれが暴走する可能性のある中、どのようなメカニズムでその暴走が防がれているのかという点。
  \item この観点での、既存論文はxxx。
\end{itemize}


\end{document}