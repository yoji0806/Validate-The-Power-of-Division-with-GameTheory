\documentclass[main.tex]{subfiles}
\begin{document}



\section{モデル}

[TODO: 画像はる。ゲームのイメージ図]

分権した国家権力機構が、全体的な方針を定めるゲームを考える。具体的には、
プレイヤーが$N=\lbrace 内閣, 国会 \rbrace$の、2期間の不完備情報動学ゲームを考える。
このゲーム上でベイジアンナッシュ均衡の条件を求めることで、各変数がどのように各権力の暴走を膨張/抑制しているのかを分析する。



司法を入れない理由は2つある。1つは、xxx。司法の違憲判決は行政執行や法律制定の数年から数十年に出される。
例として、旧優性法、足尾銅山。(TODO:他の例や、反対にすぐに政策が違憲/違法判定された例を調べる)。
その頃には、別の政権・議席割合であり、別のゲームをしていると考えるのが妥当。
2つ目は、見たかった司法の暴走の例は人質司法だが、これは政策決定とは別のゲームをしてる(司法の役割は、違法/違憲と合法/合憲の境目をはっきりさせる)。

主権者である国民(以下、"市民"とする)を入れない理由は、(TODO:結果は変わらないから国会に取り込んだ。)。


国会の効用関数を以下のように定義する。
$$u_\text{国会} = v_t(\;W\;,\; a_\text{国会}\;) × \delta^{t-1}_\text{市民}$$
$v_t(\;W\;,\; a_\text{国会}\;)$は、市民の厚生関数である。この厚生関数は、世界の状態である$W=\lbrace 有事, 平時\rbrace$と、国会が承認する政策$a_{国会}=\lbrace A, B\rbrace$によって決まる。
有事では予防的政策$A$が長期でより効果的であり、平時では即時的政策$B$が長期でより効果的であるとする。
時間割引率$\delta_{市民}=\lbrace \delta_L, \delta_H \rbrace (0\le \delta_L<\delta_H \le 1)$\footnote{TODO: この大小関係は仮定としてまとめる。}は、主権である市民が愚民($\delta_L$)であるか、賢民($\delta_H$)であるかを表している。
国会は、市民\footnote{モデル外}から直接声を聞くため、この市民の時間割引率$\delta_{市民}$を観測できるが、内閣は観測できない。

内閣の効用関数を以下のように定義する。
$$ u_\text{内閣} =\;\; \lbrace T\lbrace \alpha a_\text{国会} + \beta (1-a_\text{国会}) \rbrace  + (1-T)v_t(\;W\;,\; a_\text{国会}\;) \rbrace × \delta^{1-t}_{内閣}$$
内閣も市民の一人であるので、厚生関数$v_t(\;W\;,\; u_\text{国会}\;)$を同じように持つ。
加えて、内閣には私欲とも呼ばれる独自の正義感があり、各政策の実行自体からも効用を得る。それぞれ、予防的政策Aから得られる効用を$\alpha$, 即時的政策Bから得られる効用を$\beta$とする。
数式上では$a_{国会}=\lbrace A, B\rbrace = \lbrace 1,0\rbrace$とする。
この私欲と公共心の割合は$T:(1-T)$で表される。ただし、$T\in[0,1]$ 。
内閣は自身の私欲と公共心の割合$T:(1-T)$に加え、外交や調査機関の情報\footnote{これらはモデル外の話。}から世界の状態$W=\lbrace 有事, 平時\rbrace$を観測できるが、国会からは観測できないものとする。



\subsection{各権力の暴走の定義}
今分析の目的は、権力分立の機構が各権力の暴走をどのように防いでいるかを明らかにする、というものだ。そこで今モデル上での各権力の暴走をここで定義する。

まず、暴走の定義は「あるプレイヤーが市民の厚生関数$v_t(W, a_{国会})$の合計を最大化しようとしていない(TODO:見直す)状態」とする。
つまり、最適行動が市民の厚生関数$v_t(W, a_{国会})$の合計の最大化よりも重要なパラメータがある状態??
結果的には、市民の厚生関数$v_t(W, a_{国会})$の合計を最大化できる暴走もある。

内閣の暴走は、私欲にもとづく行政執行である。xxx

国会の暴走は、ポピュリズムである。xxx



\subsection{ゲームツリー}

\begin{figure}[htbp]
  \centering
  \includegraphics[width=1\textwidth]{./image/game_tree.jpg}
  \caption{ゲームツリー} 
  \label{fig:game_tree}
\end{figure}



ゲームは、自然が内閣と国会に私的情報を与えるところから始まる。まず自然は確率$p$で世界の状態を有事、$1-p$で平時とし、
内閣の私欲パラメーター$T\sim U(0,1)$\footnote{区間[0,1]の一様分布}を定める。これらの情報は内閣だけが観測することができる。
次に、自然は確率$q$で市民を賢民($\delta_{市民}=\delta_H$)とし、確率$1-q$で市民を愚民($\delta_{市民}=\delta_L$)とする。

内閣と国会の行動は全て1期で終わる。まず内閣は、どちらの政策が望ましいかというメッセージ$m_{内閣} = \lbrace A, B \rbrace$ を発する。
国会はそのメッセージを受けて、最終的にどちらの政策を承認するかを決定$a_{国会} = \lbrace A, B\rbrace$する。
直後、承認された政策が内閣により実行され、内閣と国会は1期目の利得を得る。

2期目は、どのプレイヤーも行動せず、実行された政策の2期目の利得を得る。




\subsection{市民の厚生関数$v_t(W, a_{国会})$における仮定}

\begin{figure}[htbp]
  \centering
  \includegraphics[width=0.7\textwidth]{./image/assumption_welfare_policy.png}
  \caption{厚生関数の仮定} 
  \label{fig:assumption_welfare_policy}
\end{figure}

有事における政策$A$によるt期の効用を$v'_{At}$とし、平時における政策$A$によるt期の効用を$v_{At}$とする。
前述の通り、有事では予防的政策$A$が長期的にはより効果的であり、平時では即時的政策$B$が長期的により効果的である。これは以下の仮定で表される。
\begin{assumption}  $v_{A1} + v_{A2} < v_{B1} + v_{B2}$ \end{assumption}
\begin{assumption}  $v'_{B1} + v'_{B2} < v'_{A1} + v'_{A2}$ \end{assumption}


有事においては、1期目は準備期間で2期目に深刻な事態が訪れるとする。
なので、平時有事に関わらず、1期目においては即時的政策$B$の効用がより大きい。これは以下の仮定で表される。
\begin{assumption}  $v_{A1} < v_{B1}$  \end{assumption}
\begin{assumption}  $v'_{A1} < v'_{B1}$ \end{assumption}

仮定2と仮定4より、以下の条件が導かれる。これは有事においては、2期目の効用は予防的政策($A$)の方が大きくなることを表す。
\begin{condition}  $v'_{B2} < v'_{A2}$ \end{condition}


上の図1はこれらの仮定を含めたものである。
このモデルの仮定を表す状況として例えば以下が考えられる。
即時的政策($B$)は予防的政策($A$)を行わない場合、と広く解釈できる。

\begin{table}[htbp]
  \caption{モデルの解釈の例}
  \label{table:data_type}
  \centering
  \begin{tabular}{cc}
    \toprule
    $W=有事$ & 予防的政策($A$) \\
    \midrule
    大地震 & 耐震工事・防波堤の建設 \\
    戦争 & 軍事費増大 \\
    少子高齢化の果て & 年金制度の改革 \\
    \bottomrule
  \end{tabular}
\end{table}



\subsection{国会の時間割引率$\delta_{市民}$に関する仮定}

\begin{figure}[htbp]
  \centering
  \includegraphics[width=0.7\textwidth]{./image/assumption_citizen_discount_rate.png}
  \caption{有事での$\delta_{市民}=\lbrace \delta_L, \delta_H \rbrace$の仮定} 
  \label{fig:assumption_citizen_discount_rate}
\end{figure}


前述の通り、$\delta_{市民}\in[0,1]$は主権者である市民が賢民($\delta_H$)か愚民($\delta_L$)かを表す。

賢民($\delta_H$)の場合は、有事では予防的政策($A$)を望むものとする。これは例えば、社会保障が増大している時に増税を受け入れるような状態である。
この仮定は数式上は以下のように表される。
\assumption{$V'_{B1} + \delta_H V'_{B2} < V'_{A1} + \delta_H V'_{A2}$}

この仮定を整理すると、
\begin{align*}
  V'_{B1} + \delta_H V'_{B2} &< V'_{A1} + \delta_H V'_{A2}\\
  V'_{B1}-V'_{A1}  &<  \delta_H (V'_{A2} - V'_{B2}) \\
  \delta_H (V'_{A2} - V'_{B2}) &> V'_{B1}-V'_{A1} \\
  \delta_H &> \frac{V'_{B1}-V'_{A1}}{V'_{A2} - V'_{B2}}
\end{align*}

よって、以下の条件を得る。
\condition{$\delta_H > \frac{V'_{B1}-V'_{A1}}{V'_{A2} - V'_{B2}}$}






\bigskip
愚民($\delta_L$)の場合は、たとえ有事であっても即時的政策($B$)を望むものとする。これは例えば、給付金によるばら撒き政策を常に望む状態である。
これは以下の仮定で表される。
\assumption{$V'_{A1} + \delta_L V'_{A2} < V'_{B1} + \delta_L V'_{B2}$}
 
この仮定を整理すると、
\begin{align*}
  V'_{A1} + \delta_L V'_{A2} &< V'_{B1} + \delta_L V'_{B2} \\
  \delta_L V'_{A2} - \delta_L V'_{B2} &< V'_{B1}  - V'_{A1} \\
  \delta_L (V'_{A2} - V'_{B2}) &< V'_{B1}  - V'_{A1}\\
  \delta_L &< \frac{V'_{B1}  - V'_{A1}}{V'_{A2} - V'_{B2}}
\end{align*}

よって、以下の条件を得る。
\condition{$\delta_L < \frac{V'_{B1}  - V'_{A1}}{V'_{A2} - V'_{B2}}$}

\bigskip
上の図2はこれらの仮定を含めたものである\footnote{TODO:$\delta_{市民} = \frac{V_{B1} - V_{A1}}{V'_{A2} - V_{B2}}$の時は、考える意味がない or 愚民か賢民の方にくっつける。}。
つまり、有事の場合では$\frac{V'_{B1} - V'_{A1}}{V'_{A2} - V'_{B2}}$を境として市民が賢民か愚民かが決まる。 


平時においては、賢民であっても愚民であっても即時的政策($B$)を望むものとする。
これは以下の仮定で表される。
\assumption{$V_{A1} + \delta_{市民} V_{A2} < V_{B1} + \delta_{市民} V_{B2} \quad{ただし}\quad 0 \le \delta_{市民}\le 1$}

これを整理して
\begin{align*}
  V_{A1} + \delta_{市民} V_{A2} < V_{B1} + \delta_{市民} V_{B2} \\
  \delta_{市民} V_{A2} - \delta_{市民} V_{B2} < V_{B1} - V_{A1} \\
  \delta_{市民} (V_{A2} - V_{B2}) < V_{B1} - V_{A1} \\
  \begin{cases}
    \delta_{市民} &< \frac{V_{B1}-V_{A1}}{V_{A2} - V_{B2}} \quad\text{if}\quad V_{B2} < V_{A2}\\
    \delta_{市民} &> \frac{V_{B1}-V_{A1}}{V_{A2} - V_{B2}} \quad\text{if}\quad V_{A2} < V_{B2}
  \end{cases}
\end{align*}

よって、賢民と愚民に場合分けした以下の条件を得る。
\condition{
\begin{align*}
  \begin{cases}
    \delta_H &< \frac{V_{B1}-V_{A1}}{V_{A2} - V_{B2}} \quad\text{if}\quad V_{B2} < V_{A2}\\
    \delta_H &> \frac{V_{B1}-V_{A1}}{V_{A2} - V_{B2}} \quad\text{if}\quad V_{A2} < V_{B2}
  \end{cases}
\end{align*}
}

\condition{
\begin{align*}
  \begin{cases}
    \delta_L &< \frac{V_{B1}-V_{A1}}{V_{A2} - V_{B2}} \quad\text{if}\quad V_{B2} < V_{A2}\\
    \delta_L &> \frac{V_{B1}-V_{A1}}{V_{A2} - V_{B2}} \quad\text{if}\quad V_{A2} < V_{B2}
  \end{cases}
\end{align*}
}
  






\theendnotes
\end{document}